%!TEX root = ../main.tex
%-------------------------------------------------------------------------------
\section{Introduction}
%-------------------------------------------------------------------------------
\noindent Structural microeconometric models clearly specify an individual's objective, the economic environment, and their institutional and informational constraints under which they operate. They are calibrated to reproduce data on observed individual individual decision and experiences. Based on the results, researchers can quantify the importance of competing economic mechanisms in determining economic outcomes and forecast the effects of policy proposals \citep{Wolpin.2013}.\\

\noindent We restrict us to the class of Eckstein-Keane-Wolpin (EKW) models \citep{Adda.2017, Blundell.2016, Keane.1997}. Labor economists often apply these models for the analysis of human capital investment decisions. Human capital is the knowledge, skills, competencies, and attributes embodied in individuals that facilitate the creation of personal, social, and economic well-being \citep{Becker.1964}. Differences in human capital attainment are a major determinant of inequality in a variety of life outcomes such as labor market success and health across and within countries \citep{OECD.2001}.

In \citet{Bhuller.2018}, for example, we apply an EKW model to analyze the mechanisms determining schooling decisions in Norway. We calibrate the model using Norwegian population panel data with nearly career-long earnings histories  After validating our model using an increase in mandatory schooling, we use it to study the underlying economic mechanisms that generate the resulting increase in average years of schooling and forecast the effect of several policy alternatives.\\

\noindent We present background material for this particular class structural economic models to facilitate transdiciplinary collaboration in their future development. We first describe the economic framework, mathematical formulation, and calibration procedure.  We then specify, simulate, and calibrate an example using our group's research codes \verb+respy+ and \verb+estimagic+. We summarize our efforts to draw on research outside economics to ameliorate the computational challenges of working with these models and explore several conceptual extensions.\\

\noindent Throughout, we only offer a limited number of seminal references and textbooks that invite further study. We introduce acronyms and symbols as needed and our notation draws form the related work by \cite{Aguirregabiria.2010}, \cite{Arcidiacono.2011}, and \cite{Puterman.1994}.
