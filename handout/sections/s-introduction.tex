%!TEX root = ../main.tex
%-------------------------------------------------------------------------------
\section{Introduction}
%-------------------------------------------------------------------------------
\paragraph{Focus of handout} This handout presents the background material for a particular class structural economic models. Its goal is to facilitate transdiciplinary collaboration in their future development by precisely outlining its economic motivation, mathematical formulation, and computational challenges for researchers outside economics.

\paragraph{Structural models}  Structural economic models clearly specify an individual's objective and the constraints of their economic environment under which they operate. They are used to quantify the importance of competing economic mechanisms in determining economic outcomes and predict the impact alternative policies before their implementation \citep{Wolpin.2013}.

\paragraph{EKW models} Our discussion is restricted to the class of Eckstein-Keane-Wolpin (EKW) models. These type of models is often used in labor economics for the analysis of human capital investment decisions. For example, \citet{Bhuller.2018} build a model to analyze the mechanisms determining schooling decisions. They focus on the option value of each level of schooling.

\paragraph{Notation} Throughout, we will introduce acronyms and symbols as needed, but a full list of both is provided in Appendix \ref{List of Symbols}. The notation draws form the related work by \cite{Puterman.1994}, \cite{Aguirregabiria.2010}, and \cite{Arcidiacono.2011}.

\paragraph{Stucture} This handout is structured as follows. We first present the economic model with a discussion of the basic economic environment. We then turn to the corresponding mathematical model and outline its solution approach. When then outline the estimation step in Section \ref{Estimation}. With this overview at hand, we discuss selected computational challenges in Section \ref{Computation}. This note concludes with an example model in Section \ref{Example}.
