%!TEX root = ../main.tex
%-------------------------------------------------------------------------------
\section{Introduction}
%-------------------------------------------------------------------------------
\paragraph{Structural models}  Structural economic models clearly specify an individual's objective and the institutional and informational constraints of their economic environment under which they operate. They are calibrated to reproduce data on observed individual individual decision and experiences. Based on the results, researchers can quantify the importance of competing economic mechanisms in determining economic outcomes and forecast the effects of alternative policies before their implementation \citep{Wolpin.2013}.

\paragraph{EKW models} We restrict us to the class of Eckstein-Keane-Wolpin (EKW) models \citep{Keane.1997,Blundell.2016,Adda.2017}. Labor economists often apply these models for the analysis of human capital investment decisions. Human capital is the knowledge, skills, competencies, and attributes embodied in individuals that facilitate the creation of personal, social, and economic well-being \citep{Becker.1964}. Differences in human capital attainment are a major determinant of inequality in a variety of life outcomes such as labor market success and health across and within countries \citep{OECD.2001}.

In \citet{Bhuller.2018}, for example, we apply an EKW model to analyze the mechanisms determining schooling decisions in Norway. We calibrate the model using Norwegian population panel data with nearly career-long earnings histories. We find an important role for the option value of schooling, which measures the value of the information generated by each additional year of schooling. After validating their model using an increase in mandatory schooling, we then use the model to study the underlying mechanisms that generate the increase of average years of schooling and forecast the effects of several policy alternatives.

\paragraph{Stucture}  We present background material for this particular class structural economic models to facilitate transdiciplinary collaboration in their future development. We first describe the economic setup, mathematical formulation, and calibration procedures. We then analyze an example application using our group's research code \verb+respy+  \citep{respy-1.0}. Finally, we review research outside economics to address computational challenges and explore possible conceptual extensions.

\paragraph{Notation} Throughout, we only offer a limited number of seminal references and textbooks to invite further study. We will introduce acronyms and symbols as needed, but a full list of both is provided in the Appendix. Our notation draws form the related work by \cite{Puterman.1994}, \cite{Aguirregabiria.2010}, and \cite{Arcidiacono.2011}.
