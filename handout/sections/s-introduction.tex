%!TEX root = ../main.tex
%-------------------------------------------------------------------------------
\section{Introduction}
%-------------------------------------------------------------------------------
\paragraph{Focus of handout} We present background material for a particular class structural economic models to facilitate transdiciplinary collaboration in their future development. We describe the economic setup, mathematical formulation, and calibration procedures for the so-called Eckstein-Keane-Wolpin (EKW) models \citep{Aguirregabiria.2010}. We provide an example application using our group's research code \verb+respy+ \citep{respy-1.0}. We draw on research outside economics to identify model components ripe for improvement and explore possible model extensions.

\paragraph{Structural models}  Structural economic models clearly specify an individual's objective and the constraints of their economic environment under which they operate. They are calibrated to reproduce data on observed individual individual behaviors. This allows to quantify the importance of competing economic mechanisms in determining economic outcomes and forecast the effects of alternative policies before their implementation \citep{Wolpin.2013}.

\paragraph{EKW models} We restrict us to the class of Eckstein-Keane-Wolpin (EKW) models. Labor economists often apply these models for the analysis of human capital investment decisions. Human capital is the knowledge, skills, competencies, and attributes embodied in individuals that facilitate the creation of personal, social, and economic well-being \citep{OECD.2001}. Differences in human capital attainment are a major determinant of inequality in a variety of life outcomes such as labor market success and health across and within countries.

In \citet{Bhuller.2018}, for example, we apply an EKW model to analyze the mechanisms determining schooling decisions in Norway. They calibrate the model using Norwegian population panel data with nearly career-long earnings histories. They find an important role for the option value of schooling, which measures the value of the information generated by each additional year of schooling. After validating their model using an increase in mandatory schooling, they then use the model to study the underlying mechanisms that generate the increase of average years of schooling and forecast the effects of several policy alternatives.

\paragraph{Stucture} This handout is structured as follows. We first present the basic setup, then offer an example implementation using the \verb+respy+ package, and finally we outline possible improvements and extensions.

\paragraph{Notation} Throughout, we only provide a very limited number of seminal references and mostly point to related textbooks instead. We will introduce acronyms and symbols as needed, but a full list of both is provided in the Appendix. The notation draws form the related work by \cite{Puterman.1994}, \cite{Aguirregabiria.2010}, and \cite{Arcidiacono.2011}.
