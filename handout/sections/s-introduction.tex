%!TEX root = ../main.tex
%-------------------------------------------------------------------------------
\section{Introduction}
%-------------------------------------------------------------------------------
\noindent Economists use structural microeconometric models to study individual decision-making. These models specify the objective of the individual, its economic environment, and the institutional and informational constraints under which it operates. Calibration of the model to observed data on individual decisions and experiences allows quantifying the importance of competing economic mechanisms in determining economic outcomes and forecasting the effects of policy proposals \citep{Wolpin.2013}.\\

\noindent We restrict our exposition to the class of Eckstein-Keane-Wolpin (EKW) models \citep{Adda.2017, Blundell.2016, Keane.1997}. Labor economists use them to study human capital investment decisions. Human capital is the knowledge, skills, competencies, and attributes embodied in individuals that facilitate the creation of personal, social, and economic well-being \citep{Becker.1964}. Differences in human capital attainment lead to inequality in a variety of life outcomes such as labor market success and health across and within countries \citep{OECD.2001}.

In \citet{Bhuller.2018}, for example, we apply an EKW model to analyze the mechanisms determining schooling decisions in Norway. We calibrate the model using Norwegian population panel data with nearly career-long earnings histories  After validating our model using a mandatory schooling reform, we gain insights into the underlying economic mechanisms that generate the effects of the policy and can forecast the effects of several policy alternatives.\\

\noindent We prepare this handout to facilitate transdisciplinary collaboration in the future development of EKW models. We proceed as follows. We first describe the economic framework, mathematical formulation, and calibration procedure. We then specify, simulate, and calibrate an example using our group's research codes \verb+respy+ and \verb+estimagic+. Finally, we summarize our efforts that draw on research outside economics to address their computational challenges and improve their usefulness.\\

\noindent Throughout, we only offer a limited number of seminal references and textbooks that invite further study. We introduce acronyms and symbols as needed, and our notation draws on the reviews by \cite{Aguirregabiria.2010}, \cite{Arcidiacono.2011}, and \cite{Puterman.1994}.
