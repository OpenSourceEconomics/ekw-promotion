% !TEX root = ../main.tex
%-------------------------------------------------------------------------------
\subsection{Model}\label{Appendix model}
%-------------------------------------------------------------------------------
We now present the exact parameterization of the per-period utility functions. The per-period utility $u_a(s_t)$ of each alternative consists of a non-pecuniary benefit $\zeta_a(s_t)$ and has, at least for the working alternatives, an additional wage component $w_a(s_t)$. Each of the components depends on the level of human capital as measured by an alternative-specific skill endowment $\bm{e_1} = \left(\bm{e}_{a,1}\right)_{a\in\mathcal{A}}$, years of completed schooling $h_t$, and occupation-specific work experience $\bm{k_t} = \left(k_{a,t}\right)_{a\in\{1, 2, 3\}}$. The per-period utility functions are also influenced by last-period choices $a_{t -1}$ and alternative-specific productivity shocks $\bm{\epsilon_t} = \left(\epsilon_{a,t}\right)_{a\in\mathcal{A}}$. Throughout alternatives the general form of the per-period utility is given by
\begin{align*}
u_a(s_t) =
\begin{cases}
    \zeta_a(\bm{k_t}, h_t, t, a_{t -1})  + w_a(\bm{k_t}, h_t, t, a_{t -1}, \bm{e}_{a,1}, \epsilon_{a,t})              & \text{if}\, a \in \{1, 2, 3\}  \\
    \zeta_a(\bm{k_t}, h_t, t, a_{t-1}, \bm{e}_{a,1}, \epsilon_{a,t})                                    & \text{\textcolor{red}{otherwise }} \text{if}\, a \in \{4, 5\}.
\end{cases}
\end{align*}
Individuals enter the model as a particular type $j \in \mathcal{J} = \{1, 2, 3, 4\}$ which is reflected through an initial skill endowment $\left(\bm{e}_{a,1}\right)_{a\in\mathcal{A}} = \left(e_{j,a,1}\right)_{a\in\mathcal{A}, j \in \mathcal{J}}$. Individuals are homogeneous with respect to their per-period utility functions but heterogeneous with respect to their initial skill endowments. 
\vspace{20pt} \\
$\left(\bm{e}_{a,1}\right)_{a\in\mathcal{A}} = \Big( \bm{e}_{j,a,1} \Big)_{\substack{a\in\mathcal{A}, \\ j \in \mathcal{J} }}$.

%-------------------------------------------------------------------------------
\subsection{Utility for working alternatives}
%-------------------------------------------------------------------------------
The per-period utility for working alternatives is the sum of a non-pecuniary benefit and a wage component. The wage $w_{a}(s_t)$ is given by the product of the market-equilibrium rental price $\rho_{a,t}$ and an occupation-specific skill level $x_{a}(s_t)$. This specification leads to a standard logarithmic wage equation in which the constant term is given as sum of the skill rental price $\ln(r)$ and the initial skill endowment $\bm{e}_{a,1}$.\\

The occupation-specific skill level is determined by a skill production function, which differs for civilian and military occupations. The general form of the skill production function includes a deterministic component $\Gamma_a$ and a multiplicative stochastic productivity shock $\epsilon_{a,t}$
%
\begin{align}\label{eq:OccupationSpecificSkillLevel}
    x_{a}(\bm{k_t}, h_t, t, a_{t-1}, \bm{e}_{a, 1}, \epsilon_{a,t}) & = \exp \big( \Gamma_{a}(\bm{k_t},  h_t, t, a_{t-1}, \bm{e}_{a,1}) \cdot \epsilon_{a,t} \big) \nonumber \\
                & = \exp \big( \Gamma_a(\bm{k_t},  h_t, t, a_{t-1}, \bm{e}_{a,1}) \big) \cdot \exp \big( \epsilon_{a,t} \big).
\end{align}
%-------------------------------------------------------------------------------
\subsubsection{Utility for civilian occupations}
%-------------------------------------------------------------------------------
We now state the parameterization for the civilian occupations, we provide some economic reasoning behind their specification for the white collar occupation, the blue collar is defined analogously.

\paragraph{White-collar occupation} Equation (\ref{eq:SkillLevelWhiteCollar}) provides the parameterization of the deterministic skill-production component in the white-collar occupation.
%
\begin{align}\label{eq:SkillLevelWhiteCollar}
    \Gamma_1(\bm{k_t}, h_t, t, a_{t-1}, e_{1,j}) = e_{1,j} & + \beta_{1,1} \cdot h_t + \beta_{1, 2} \cdot \ind[h_t \geq 12] + \beta_{1,3} \cdot \ind[h_t\geq 16] \nonumber\\
                                 \nonumber & + \gamma_{1, 2} \cdot  k_{1,t} + \gamma_{1,3} \cdot  (k_{1,t})^2 + \gamma_{1,4} \cdot  \ind[k_{1,t} > 0] \\
                                  \nonumber & + \gamma_{1,5} \cdot  t + \gamma_{1,6} \cdot \ind[t < 18] \nonumber\\
                                  & + \gamma_{1,7} \cdot  a_{t-1} + \gamma_{1,8} \cdot  k_{2,t} + \gamma_{1,9} \cdot  k_{3,t}.
\end{align}
%

given by years of completed schooling ($\beta_{1,1}$), graduation effects ($\beta_{1,2}, \beta_{1,3}$), occupation-specific experience ($\gamma_{1,2}, \gamma_{1,3},  \gamma_{1,8}, \gamma_{1,9}$), skill depreciation ($\gamma_{1,7}$), and age effects like being a minor ($\gamma_{1,5}, \gamma_{1,6}$). The coefficient $\gamma_{1,4}$ captures whether an individual has ever worked in the occupation before.\\

Equation (\ref{eq:NonWageWhiteCollar}) provides the parameterization for the non-wage component in the white-collar occupation.
%
\begin{align}\label{eq:NonWageWhiteCollar}
\zeta_{1}(\bm{k_t}, h_t, a_{t-1}) = &~c_{1,1} \cdot \ind[a_{t-1} \neq 1] + c_{1,2} \cdot \ind[k_{1,t} = 0] \nonumber \\
                            & + \alpha_1 \nonumber \\
                            & + \vartheta_1 \cdot \ind[h_t \geq 12] + \vartheta_2 \cdot \ind[h_t \geq 16] + \vartheta_3 \cdot \ind[k_{3,t} = 1].
\end{align}
%
The non-wage component includes mobility and search costs, which are higher for individuals who never worked in the occupation ($c_{1,1} < 0, c_{1,2}< 0)$. A constant $\alpha_1$ captures the net monetary-equivalent value of working conditions (either negative or positive). Adding coefficients for graduation and military effects ($\vartheta_{1}, \vartheta_{2}, \vartheta_{3}$).

\paragraph{Blue-collar occupation} The deterministic component and the nonpecuniary element in the blue-collar occupation follow the same line of reasoning.
%
\begin{align*} 
    \Gamma_2(\bm{k_t}, h_t, t, a_{t-1}, e_{2,j}) = e_{2,j} & + \beta_{2,1} \cdot h_t + \beta_{2, 2} \cdot \ind[h_t \geq 12] + \beta_{2,3} \cdot \ind[h_t\geq 16] \\
    							 & + \gamma_{2, 2} \cdot  k_{2,t} + \gamma_{2,3} \cdot  (k_{2,t})^2 + \gamma_{2,4} \cdot  \ind[k_{2,t} > 0] \\
                                   & + \gamma_{2,5} \cdot  t + \gamma_{2,6} \cdot \ind[t < 18] \\
                                  & + \gamma_{2,7} \cdot  a_{t-1} + \gamma_{2,8} \cdot  k_{1,t} + \gamma_{2,9} \cdot  k_{3,t}. \\
                                  &\\
\zeta_{2}( \bm{k_t}, h_t, a_{t-1} ) = &~c_{2,1} \cdot \ind[a_{t-1} \neq 1] + c_{2,2} \cdot \ind[k_{2,t} = 0] \\
                            & + \alpha_2 \\
                            & + \vartheta_1 \cdot \ind[h_t \geq 12] + \vartheta_2 \cdot \ind[h_t \geq 16] + \vartheta_3 \cdot \ind[k_{3,t} = 1].
\end{align*}


%-------------------------------------------------------------------------------
\FloatBarrier\subsubsection{Military}
%-------------------------------------------------------------------------------
The reward for serving in the military differs in its structure from the civilian sector. The monetary-valued constant $\alpha_{3,t}$ enters in a multiplicative fashion and depends on the age of the individual.
%
\begin{align}\label{eq:RewardMilitary}
    u_{3}(s_t) = \zeta_3(\bm{k_t}, h_t, t , a_{t -1})  + \exp \big( \alpha_{3, t} \big) \cdot w_{3,t}
\end{align}

Still, the wage $w_{3}$ is given as the product of the skill-rental price and mility skills.
%
\begin{align}
    \Gamma_3( k_{3,t}, h_t, t, e_3) = e_3 & + \beta_{3,1} \cdot h_t \nonumber \\
	               \nonumber &+ \gamma_{3,2} \cdot  k_{3,t} + \gamma_{3,3} \cdot (k_{3,t})^2 + \gamma_{3,4} \cdot \ind[k_{3,t} < 0] \\
									 & + \gamma_{3,5} \cdot t + \gamma_{3,6} \cdot \ind[t < 18].
\end{align}
%
All types $j = 1, \dots, J$ are endowed with the same base skill endowment, i.e. $e_3 = e_{3,j} = \dots = e_{3, J}$. Neither a finished degree ($\beta_{3,2} = 0, \beta_{3,3} = 0$) nor experience in any of the civilian sectors ($\gamma_{3,8} = 0$) add to the military-specific skill level. There is no depreciation of military-specific skills ($\gamma_{3,7} = 0$). The deterministic component is thus given by
%
The nonpecuniary benefit is specified as follows:
%
\begin{align*}
\zeta_{3}( k_{3.t}, h_t, a_{t-1} )  & =   + c_{3,2} \cdot \ind[k_{3,t} = 0] \nonumber \\
  & + \vartheta_1 \cdot \ind[h_t \geq 12] + \vartheta_2 \cdot \ind[h_t \geq 16]
\end{align*}

There is no penalty for leaving the military prematurely ($\vartheta_3 = 0$), and no effect for never having served in the military before ($\gamma_{3,7} = 0$).\\

\autoref{tab:ModelParameters} summarizes the parameters prevalent in the military occupation.


% ------------------------------------------------------------------------------
\FloatBarrier\subsection{Utility for Non-working alternatives}
% ------------------------------------------------------------------------------
The most distinctive feature towards the occupational reward function concerns the stochastic component which enters additively instead of multiplicatively. There is no wage component.
%-------------------------------------------------------------------------------
\FloatBarrier\subsubsection{School}
%-------------------------------------------------------------------------------
Equation \ref{eq:RewardSchooling} shows the parameterization of the non-pecuniary benefits.
%
\begin{align}\label{eq:RewardSchooling}
	u_{4}(s_t) = e_{4,j} & + \beta_{tc_1} \cdot \ind[h_t \geq 12] + \beta_{tc_2} \cdot \ind[h_t \geq 16] 														\nonumber \\
    							  & + \beta_{rc_1} \cdot \ind[a_{t-1} \neq 1, h_t < 12] + \beta_{rc_2} \cdot \ind[a_{t-1} \neq 1, h_t \geq 12] 			   \nonumber \\
    							  & + \gamma_{4,5} \cdot t + \gamma_{4,6} \cdot \ind[t < 18] 																					  \nonumber \\
     							  & + \vartheta_1 \cdot \ind[h_t \geq 12] + \vartheta_2 \cdot \ind[h_t \geq 16] + \vartheta_3 \cdot \ind[k_{3,t} = 1) \nonumber \\
      							  & + \epsilon_{4,t}.
\end{align}
%
Type-heterogeneity in endowments $e_{4,j}$ translates into heterogeneity in utility derived from schooling. Attending highschool or college involves direct tuition costs ($\beta_{tc_1} < 0, \beta_{tc_2} <0$). Individual who return to school incur re-entry costs ($\beta_{rc_1} <0 , \beta_{rc_2} < 0$).

Table \ref{tab:ModelParameters} summarizes the parameters prevalent in the school sector.


%-------------------------------------------------------------------------------
\FloatBarrier\subsubsection{Home}
%-------------------------------------------------------------------------------
The reward for staying at home only depends on the age of the individual
%
\begin{align}
    u_5(s_t) =  e_{5,j} & + \beta_{5,2} \cdot \ind[h_t \geq 12] + \beta_{5,3} \cdot \ind[h_t \geq 16] 			 \nonumber \\
    							   & + \gamma_{5,5} \cdot \ind[18 \leq t \leq 20] + \gamma_{5,6} \cdot \ind[t \geq 21] \nonumber \\
    							   & + \vartheta_3 \cdot \ind[k_{3,t} = 1]  \nonumber \\
    							   & + \epsilon_{5,t}.
\end{align}
%
\autoref{tab:ModelParameters} summarizes the parameters prevalent in the home sector.

\clearpage 

%-------------------------------------------------------------------------------
\FloatBarrier\subsection{Overview parameters}
%-------------------------------------------------------------------------------

\begin{ThreePartTable}
% Information available at https://ftp.agdsn.de/pub/mirrors/latex/dante/macros/latex/contrib/threeparttablex/threeparttablex.pdf

\begin{TableNotes}
	\item \textbf{Note:} The listed parameters are represented as an overview. The per-period utilities for the alternatives do not necessarily include all of them.
\end{TableNotes}

\begin{longtable}{@{}cll@{}}
\caption{Overview of parameters in the \citet{Keane.1997} extended model.}
\label{tab:ModelParameters}

\setlength\extrarowheight{2.5pt}

% Settings longtable
\\
\toprule 
\textbf{Parameter}            &  &  \multicolumn{1}{l}{\textbf{Description}}              \\ \midrule 
\endfirsthead 

% Alternative 1 header for the beginning on next page
%\midrule
%Parameter            &  &  \multicolumn{1}{l}{Description}     \\ \midrule

% Alternative 2 header for the beginning on next page
\multicolumn{3}{c}{\tikz\draw [thick,dash dot] (0,0) -- (15,0);} \vspace{-5pt} \\
\multicolumn{3}{c}{continued from previous page} \vspace{-10pt} \\
\multicolumn{3}{c}{\tikz\draw [thick,dash dot] (0,0) -- (15,0);} \\
\endhead 

\multicolumn{3}{c}{\tikz\draw [thick,dash dot] (0,0) -- (15,0);} \vspace{-5pt} \\
\multicolumn{3}{c}{continued on next page } \vspace{-10pt} \\
\multicolumn{3}{c}{\tikz\draw [thick,dash dot] (0,0) -- (15,0);} \\
\endfoot

\bottomrule 
\insertTableNotes
\endlastfoot 

% Start table
\midrule
\multicolumn{3}{l}{Preference and type-specific parameters}																		 \\ \midrule 
$\delta$ 				&  & discount factor																			  \\
$e_{j, a}$			&  & initial endowment of type $j$ in alternative $a$ specific skills 	  \\ [7.5pt] \midrule 
\multicolumn{3}{l}{Common parameters per-period utility}												\\ \midrule 
$\alpha_a$           &  & return non-wage working conditions		   \\
$\vartheta_1$        &  & non-pecuniary premium of finishing high-school                 								    \\
$\vartheta_2$        &  & non-pecuniary premium finishing college															    \\
$\vartheta_3$        &  & non-pecuniary premium of leaving military early						  \\[7.5pt] \midrule 
\multicolumn{3}{l}{Schooling-related parameters}															   \\ \midrule
% Work
$\beta_{a,1}$        &  & return additional year of completed schooling 								\\
$\beta_{a,2}$        &  & skill premium high-school graduate										      \\
$\beta_{a,3}$        &  & skill premium college graduate													   	\\
% Military
% School
$\beta_{tc_1}$       &  & tuition cost high-school                      											\\
$\beta_{tc_2}$       &  & tuition cost college                          												\\
$\beta_{rc_1}$       &  & re-entry cost high-school                     										   \\
$\beta_{rc_2}$       &  & re-entry cost college                        												   \\
% Home
$\beta_{5,2}$        &  & skill premium high-school graduate            									\\
$\beta_{5,3}$        &  & skill premium college graduate                									     \\ [7.5pt] \midrule
\multicolumn{3}{l}{Experience-related parameters}           													 \\
\midrule 
$\gamma_{a,1}$       &  & return same-sector experience                 									 \\
$\gamma_{a,2}$       &  & return squared same-sector experience         								\\
$\gamma_{a,3}$       &  & premium having worked in sector before        							   \\
$\gamma_{a,4}$       &  & return age effect                             											     \\
$\gamma_{a,5}$       &  & return age effect being a minor               										\\
$\gamma_{a,6}$       &  & premium remaining in same sector              								   \\
$\gamma_{a,7}$       &  & return civilian cross-sector experience       								    \\
$\gamma_{a,8}$       &  & return non-civilian sector experience       										 \\
$\gamma_{3,1}$       &  & return same-sector experience                 									  \\
$\gamma_{3,2}$       &  & return squared same-sector experience    										 \\
$\gamma_{3,3}$       &  & premium having worked in sector before   										\\
$\gamma_{3,4}$       &  & return age effect                             												 \\
$\gamma_{3,5}$       &  & return age effect being a minor              	   										\\
$\gamma_{4,4}$       &  & return age effect                             												 \\
$\gamma_{4,5}$       &  & return age effect being a minor                  										\\
$\gamma_{5,4}$       &  & return age between 17 and 21                 	  									   \\
$\gamma_{5,5}$       &  & return age older than 20							   										\\[7.5pt] \midrule
\multicolumn{3}{l}{Mobility and search parameters}          													  \\ \midrule 
$c_{a,1}$            &  & premium switching to occupation $a$           									   \\
$c_{a,2}$            &  & premium for working first time in occupation $a$         										  \\
$c_{3,2}$            &  & premium for serving first time in military    										  \\[7.5pt] \midrule
\multicolumn{3}{l}{Error correlation}          													  									\\ \midrule 
$\sigma_{a,a}$	&	& standard deviation of shock in alternative $a$									\\
$\sigma_{i,j}$ &	& correlation between shocks of alternative $a = i$ and $a=j$ with $i \neq j$ \\

\end{longtable}
\end{ThreePartTable}
