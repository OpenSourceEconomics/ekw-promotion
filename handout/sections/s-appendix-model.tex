% !TEX root = ../main.tex
%-------------------------------------------------------------------------------
\subsection{Model}\label{Appendix model}
%-------------------------------------------------------------------------------
We now present the exact parameterization of the per-period utility functions. The per-period utility $u_a(s_t)$ of each alternative consists of a non-pecuniary benefit $\zeta_a(s_t)$ and has, at least for the working alternatives, an additional wage component $w_t(s_t)$. Each components depends on the level of human capital as measured by an alternative-specific skill endowment $\bm{e_1} = \left(e_{a,1}\right)_{a\in\mathcal{A}}$, years of completed schooling $h_t$, and occupation-specific work experience $\bm{k_t} = \left(k_{a,t}\right)_{a\in\{1, 2, 3\}}$. There is also an impact of last-period choice $a_{t -1}$ and alternative-specific productivity shocks
$\bm{\epsilon_t} = \left(\epsilon_{a,t}\right)_{a\in\mathcal{A}}$.

\begin{align*}
u_a(s_t) =
\begin{cases}
    \zeta_a(t, h_t, \bm{k_t}, a_{t -1})  + w_a(e_{a,1}, t, h_t, \bm{k_t}, a_{t -1}, \epsilon_{a,t})              & \text{if}\, a \in [1, 2, 3]  \\
    \zeta_a(e_{a,1}, t, h_t, \bm{k_t}, a_{t -1}, \epsilon_{a,t})                                    & \text{otherwise}
\end{cases}
\end{align*}
%-------------------------------------------------------------------------------
\subsection{Utility for working alternatives}
%-------------------------------------------------------------------------------
The wage $w_{a,t}$ is determined by the product of the market-equilibrium rental price $\rho_{a,t}$ and the occupation-specific skill level $x_{a,t}$. This specification leads to a standard logarithmic wage equation in which the constant term is the sum of the skill rental price $\ln(r)$ and the initial skill endowment $e_{a,1}$.\\

The occupation-specific skill level is determined by a skill production function, which includes a deterministic component $\Gamma_a$ and a multiplicative stochastic productivity shock $\epsilon_{1,a}$
%
\begin{align}\label{eq:WhiteCollarSkillLevel}
    x_{1,t}(\bm{k_t}, h_t, t, a_{t-1}, e_{1,j}, \epsilon_{1,t}) & = \exp \big( \Gamma_{1}(\bm{k_t},  h_t, t, a_{t-1}, e_{1,j}) \cdot \epsilon_{1,t} \big) \\\nonumber
                & = \exp \big( \Gamma_1(\bm{k_t},  h_t, t, a_{t-1}, e_{1,j}) \big) \cdot \exp \big( \epsilon_{1,t} \big)
\end{align}
%-------------------------------------------------------------------------------
\subsubsection{Utility for civilian occupations}
%-------------------------------------------------------------------------------
We now state the parameterization for the civilian occupations, we provide some economic reasoning behind their specification for the white collar occupation, the blue collar is defined analogously.

\paragraph{White-collar occupation} Equation (\ref{eq:SkillLevelWhiteCollar}) provides the parameterization of the deterministic skill-production component in the white-collar occupation.
%
\begin{align}\label{eq:SkillLevelWhiteCollar}
    \Gamma_1(\bm{k_t}, h_t, t, a_{t-1}, e_{1,j}) = e_{1,j} & + \beta_{1,1} \cdot h_t + \beta_{1, 2} \cdot \ind[h_t \geq 12] + \beta_{1,3} \cdot \ind[h_t\geq 16] \nonumber\\
                                 \nonumber & + \gamma_{1, 2} \cdot  k_{1,t} + \gamma_{1,3} \cdot  (k_{1,t})^2 + \gamma_{1,4} \cdot  \ind[k_{1,t} > 0] \\
                                  \nonumber & + \gamma_{1,5} \cdot  t + \gamma_{1,6} \cdot \ind[t < 18] \nonumber\\
                                  & + \gamma_{1,7} \cdot  a_{t-1} + \gamma_{1,8} \cdot  k_{2,t} + \gamma_{1,9} \cdot  k_{3,t}.
\end{align}
%

given by years of completed schooling ($\beta_{1,1}$), graduation effects ($\beta_{1,2}, \beta_{1,3}$), occupation-specific experience ($\gamma_{1,2}, \gamma_{1,3},  \gamma_{1,8}, \gamma_{1,9}$), skill depreciation ($\gamma_{1,7}$), and age effects like being a minor ($\gamma_{1,5}, \gamma_{1,6}$). The coefficient $\gamma_{1,4}$ captures whether an individual has ever worked in the occupation before.\\

Equation (\ref{eq:NonWageWhiteCollar}) provides the parameterization for the non-wage component in the white-collar occupation.
%
\begin{align}\label{eq:NonWageWhiteCollar}
\zeta_{1}(\bm{k_t}, h_t, a_{t-1}) = &~c_{1,1} \cdot \ind[a_{t-1} \neq 1] + c_{1,2} \cdot \ind[k_{1,t} = 0] \nonumber \\
                            & + \alpha_1 \nonumber \\
                            & + \vartheta_1 \cdot \ind[h_t \geq 12] + \vartheta_2 \cdot \ind[h_t \geq 16] + \vartheta_3 \cdot \ind[k_{3,t} = 1].
\end{align}
%
The non-wage component includes mobility and search costs, which are higher for individuals who never worked in the occupation ($c_{1,1} < 0, c_{1,2}< 0)$. A constant $\alpha_1$ captures the net monetary-equivalent value of working conditions (either negative or positive). Adding coefficients for graduation and military effects ($\vartheta_{1}, \vartheta_{2}, \vartheta_{3}$).

\paragraph{Blue-collar occupation} The deterministic component and the nonpecuniary element in the blue-collar occupation follow the same line of reasoning.
%
\begin{align*} 
    \Gamma_2(\bm{k_t}, h_t, t, a_{t-1}, e_{2,j}) = e_{2,j} & + \beta_{2,1} \cdot h_t + \beta_{2, 2} \cdot \ind[h_t \geq 12] + \beta_{2,3} \cdot \ind[h_t\geq 16] \\
    							 & + \gamma_{2, 2} \cdot  k_{2,t} + \gamma_{2,3} \cdot  (k_{2,t})^2 + \gamma_{2,4} \cdot  \ind[k_{2,t} > 0] \\
                                   & + \gamma_{2,5} \cdot  t + \gamma_{2,6} \cdot \ind[t < 18] \\
                                  & + \gamma_{2,7} \cdot  a_{t-1} + \gamma_{2,8} \cdot  k_{1,t} + \gamma_{2,9} \cdot  k_{3,t}. \\
                                  &\\
\zeta_{2}( \bm{k_t}, h_t, a_{t-1} ) = &~c_{2,1} \cdot \ind[a_{t-1} \neq 1] + c_{2,2} \cdot \ind[k_{2,t} = 0] \\
                            & + \alpha_2 \\
                            & + \vartheta_1 \cdot \ind[h_t \geq 12] + \vartheta_2 \cdot \ind[h_t \geq 16] + \vartheta_3 \cdot \ind[k_{3,t} = 1].
\end{align*}


\autoref{tab:ParametersCivilianOccupations} summarizes the parameterization in the civilian occupations.
\begin{table}[h]
\caption{Summary parameters civilian occupation $a \in \{1,2\}$.}
\label{tab:ParametersCivilianOccupations}
\vspace{5pt}
\centering
\setlength\extrarowheight{2.5pt}
\begin{tabular}{@{}cll@{}}
\toprule
Parameter            &  &  \multicolumn{1}{l}{Description}              \\ \midrule
\multicolumn{3}{l}{\underline{Schooling-related parameters}}            \\[5pt]
$\beta_{a,1}$        &  & return additional year of completed schooling \\
$\beta_{a,2}$        &  & skill premium high-school graduate            \\
$\beta_{a,3}$        &  & skill premium college graduate                \\[7.5pt]
\multicolumn{3}{l}{\underline{Experience-related parameters}}           \\[5pt]
$\gamma_{a,2}$       &  & return same-sector experience                 \\
$\gamma_{a,3}$       &  & return squared same-sector experience         \\
$\gamma_{a,4}$       &  & premium having worked in sector before        \\
$\gamma_{a,5}$       &  & return age effect                             \\
$\gamma_{a,6}$       &  & return age effect being a minor               \\
$\gamma_{a,7}$       &  & premium remaining in same sector              \\
$\gamma_{a,8}$       &  & return civilian cross-sector experience       \\
$\gamma_{a,9}$       &  & return non-civilian sector experience         \\[7.5pt]
\multicolumn{3}{l}{\underline{Mobility and search parameters}}          \\[5pt]
$c_{a,1}$            &  & premium switching to occupation $a$           \\
$c_{a,2}$            &  & premium for working first time in $a$         \\[7.5pt]
\multicolumn{3}{l}{\underline{Common parameters}}                    \\[5pt]
$\alpha_a$           &  & return non-wage working conditions            \\
$\vartheta_1$        &  & premium finishing high-school                 \\
$\vartheta_2$        &  & premium finishing college                     \\
$\vartheta_3$        &  & premium military                              \\[7.5pt]
\bottomrule
\end{tabular}
\end{table}


%-------------------------------------------------------------------------------
\FloatBarrier\subsubsection{Military}
%-------------------------------------------------------------------------------
The reward for serving in the military differs in its structure from the civilian sector. The monetary-valued constant $\alpha_{3,t}$ enters in a multiplicative fashion and depends on the age of the individual.
%
\begin{align}\label{eq:RewardMilitary}
    u_{3}(s_t) = \zeta_3(t, h_t, \bm{k_t}, a_{t -1})  + \exp \big( \alpha_{3, t} \big) \cdot w_{3,t}
\end{align}

Still, the wage $w_{3,t}$ is given as the product of the skill-rental price and mility skills.
%
\begin{align}
    \Gamma_3( k_{3,t}, h_t, t, e_3) = e_3 & + \beta_{3,1} \cdot h_t \nonumber \\
	               \nonumber &+ \gamma_{3,2} \cdot  k_{3,t} + \gamma_{3,3} \cdot (k_{3,t})^2 + \gamma_{3,4} \cdot \ind[k_{3,t} < 0] \\
									 & + \gamma_{3,5} \cdot t + \gamma_{3,6} \cdot \ind[t < 18].
\end{align}
%
All types $j = 1, \dots, J$ are endowed with the same base skill endowment, i.e. $e_3 = e_{3,j} = \dots = e_{3, J}$. Neither a finished degree ($\beta_{3,2} = 0, \beta_{3,3} = 0$) nor experience in any of the civilian sectors ($\gamma_{3,8} = 0$) add to the military-specific skill level. There is no depreciation of military-specific skills ($\gamma_{3,7} = 0$). The deterministic component is thus given by
%
The nonpecuniary benefit is specified as follows:
%
\begin{align*}
\zeta_{3}( \bm{k_t}, h_t, a_{t-1} )  & =   + c_{3,2} \cdot \ind[k_{3,t} = 0] \nonumber \\
  & + \vartheta_1 \cdot \ind[h_t \geq 12] + \vartheta_2 \cdot \ind[h_t \geq 16]
\end{align*}

There is no penalty for leaving the military prematurely ($\vartheta_3 = 0$), and no effect for never having served in the military before ($\gamma_{3,7} = 0$).\\

\autoref{tab:ParametersMilitaryOccupation} summarizes the parameters prevalent in the military occupation.


\begin{table}[hbt!]
\caption{Summary parameters military occupation.}
\label{tab:ParametersMilitaryOccupation}
\vspace{5pt}
\centering
\setlength\extrarowheight{2.5pt}
\begin{tabular}{@{}cll@{}}
\toprule
Parameter            &  &  \multicolumn{1}{l}{Description}              \\ \midrule
\multicolumn{3}{l}{\underline{Schooling-related parameters}}            \\[5pt]
$\beta_{3,1}$        &  & return additional year of completed schooling \\[7.5pt]
\multicolumn{3}{l}{\underline{Experience-related parameters}}           \\[5pt]
$\gamma_{3,2}$       &  & return same-sector experience                 \\
$\gamma_{3,3}$       &  & return squared same-sector experience         \\
$\gamma_{3,4}$       &  & premium having worked in sector before        \\
$\gamma_{3,5}$       &  & return age effect                             \\
$\gamma_{3,6}$       &  & return age effect being a minor               \\[7.5pt]
\multicolumn{3}{l}{\underline{Mobility and search parameters}}          \\[5pt]
$c_{3,2}$            &  & premium for working first time in $a$         \\[7.5pt]
\multicolumn{3}{l}{\underline{Common parameters}}                  \\[5pt]
$\alpha_{3,t}$        &  & return non-wage working conditions            \\
$\vartheta_1$        &  & premium finishing high-school                 \\
$\vartheta_2$        &  & premium finishing college                     \\[7.5pt]
\bottomrule
\end{tabular}
\end{table}

% ------------------------------------------------------------------------------
\FloatBarrier\subsection*{Non-working alternatives}
% ------------------------------------------------------------------------------
The most distinctive feature towards the occupational reward function concerns the stochastic component which enters additively instead of multiplicatively. There is no wage component.
%-------------------------------------------------------------------------------
\FloatBarrier\subsubsection{School}
%-------------------------------------------------------------------------------
Equation \ref{eq:RewardSchooling} shows the parameterization of the non-pecuniary benefits.
%
\begin{align}\label{eq:RewardSchooling}
	u_{4}(s_t) = e_{4,j} & + \beta_{tc_1} \cdot \ind[h_t \geq 12] + \beta_{tc_2} \cdot \ind[h_t \geq 16] 														\nonumber \\
    							  & + \beta_{rc_1} \cdot \ind[a_{t-1} \neq 1, h_t < 12] + \beta_{rc_2} \cdot \ind[a_{t-1} \neq 1, h_t \geq 12] 			   \nonumber \\
    							  & + \gamma_{4,5} \cdot t + \gamma_{4,6} \cdot \ind[t < 18] 																					  \nonumber \\
     							  & + \vartheta_1 \cdot \ind[h_t \geq 12] + \vartheta_2 \cdot \ind[h_t \geq 16] + \vartheta_3 \cdot \ind[k_{3,t} = 1) \nonumber \\
      							  & + \epsilon_{4,t}.
\end{align}
%
Type-heterogeneity in endowments $e_{4,j}$ translates into heterogeneity in utility derived from schooling. Attending highschool or college involves direct tuition costs ($\beta_{tc_1} < 0, \beta_{tc_2} <0$). Individual who return to school incur re-entry costs ($\beta_{rc_1} <0 , \beta_{rc_2} < 0$).

Table \ref{tab:ParametersSchooling} summarizes the parameters prevalent in the school sector.

\begin{table}[h]
\caption{Summary parameters schooling.}
\label{tab:ParametersSchooling}
\vspace{5pt}
\centering
\setlength\extrarowheight{2.5pt}
\begin{tabular}{@{}cll@{}}
\toprule
Parameter            &  &  \multicolumn{1}{l}{Description}              \\ \midrule
\multicolumn{3}{l}{\underline{Schooling-related parameters}}            \\[5pt]
$\beta_{tc_1}$       &  & tuition cost high-school                      \\
$\beta_{tc_2}$       &  & tuition cost college                          \\
$\beta_{rc_1}$       &  & re-entry cost high-school                     \\
$\beta_{rc_2}$       &  & re-entry cost college                         \\[7.5pt]
\multicolumn{3}{l}{\underline{Experience-related parameters}}           \\[5pt]
$\gamma_{4,5}$       &  & return age effect                             \\
$\gamma_{4,6}$       &  & return age effect being a minor               \\[7.5pt]
\multicolumn{3}{l}{\underline{Common parameters}}                 \\[5pt]
$\vartheta_1$        &  & premium finishing high-school                 \\
$\vartheta_2$        &  & premium finishing college                     \\
$\vartheta_3$        &  & premium military                              \\[7.5pt]
\bottomrule
\end{tabular}
\end{table}


%-------------------------------------------------------------------------------
\FloatBarrier\subsubsection{Home}
%-------------------------------------------------------------------------------
The reward for staying at home only depends on the age of the individual
%
\begin{align}
    u_5(s_t) =  e_{5,j} & + \beta_{5,2} \cdot \ind[h_t \geq 12] + \beta_{5,3} \cdot \ind[h_t \geq 16] 			 \nonumber \\
    							   & + \gamma_{5,5} \cdot \ind[18 \leq t \leq 20] + \gamma_{5,6} \cdot \ind[t \geq 21] \nonumber \\
    							   & + \vartheta_3 \cdot \ind[k_{3,t} = 1]  \nonumber \\
    							   & + \epsilon_{5,t}.
\end{align}
%
\autoref{tab:ParametersHome} summarizes the parameters prevalent in the home sector.

\begin{table}[h]
\caption{Summary parameters home sector.}
\label{tab:ParametersHome}
\vspace{5pt}
\centering
\setlength\extrarowheight{2.5pt}
\begin{tabular}{@{}cll@{}}
\toprule
Parameter            &  &  \multicolumn{1}{l}{Description}              \\ \midrule
\multicolumn{3}{l}{\underline{Schooling-related parameters}}            \\[5pt]
$\beta_{5,2}$        &  & skill premium high-school graduate            \\
$\beta_{5,3}$        &  & skill premium college graduate                \\[7.5pt]
\multicolumn{3}{l}{\underline{Experience-related parameters}}           \\[5pt]
$\gamma_{5,5}$       &  & return age between 17 and 21                  \\
$\gamma_{5,6}$       &  & return age older than 20                      \\[7.5pt]
\multicolumn{3}{l}{\underline{Common parameters}}                \\[5pt]
$\vartheta_3$        &  & premium military                              \\[7.5pt]
\bottomrule
\end{tabular}
\end{table}

