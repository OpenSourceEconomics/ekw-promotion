\documentclass[a4paper,12pt,bold]{scrartcl}

\usepackage{color,colortbl}																			%Farbige Tabellen
\usepackage{xcolor}

\usepackage{apacite}
\usepackage{footnote,placeins}
\makesavenoteenv{tabular}
%\usepackage{enumitem}
\usepackage{footmisc}
\newcommand*\diff{\mathop{}\!\mathrm{d}}
\renewcommand{\baselinestretch}{1.3}\normalsize
\newcommand{\vect}[1]{\mathbf{#1}}
\newcommand{\thin}{\thinspace}
\newcommand{\thick}{\thickspace}
\newcommand{\N}{\mathcal{N}}	%Normal Distribution
\newcommand{\U}{\mathrm{U}}	%Uniform Distribution
\newcommand{\D}{\mathrm{D}}	%Dirichlet Distribution
\newcommand{\W}{\mathrm{W}}	%Wishart Distribution
\newcommand{\E}{\mathrm{E}}		%Expectation
\newcommand{\Ind}{\mathbb{I}\,}	%Indicator Function

\newcommand{\bs}{\boldsymbol}
\newcommand{\var}{\mathrm{var}\thin}
\newcommand{\plim}{\mathrm{plim}\thin}
\newcommand{\cov}{\mathrm{cov}\thin}
\newcommand\indep{\protect\mathpalette{\protect\independenT}{\perp}}
\def\independenT#1#2{\mathrel{\rlap{$#1#2$}\mkern5mu{#1#2}}}
\usepackage{bbm}
%\usepackage{endfloat}
\renewcommand{\vec}[1]{\mathbf{#1}}

\usepackage{algpseudocode,tabularx,ragged2e}
\newcolumntype{C}{>{\centering\arraybackslash}X} % centered "X" column
\newcolumntype{L}{>{\arraybackslash}X} % centered "X" column

\usepackage{algorithmicx}

\usepackage{algorithm}

\let\Algorithm\algorithm
\renewcommand\algorithm[1][]{\Algorithm[#1]\setstretch{1.5}}


\definecolor{lightgrey}{gray}{0.90}	%Farben mischen
\definecolor{grey}{gray}{0.85}
\definecolor{darkgrey}{gray}{0.65}
\definecolor{lightblue}{rgb}{0.8,0.85,1}

\newcolumntype{g}{>{\columncolor{gray}}c}

\usepackage{booktabs}
\usepackage{epigraph}
\usepackage[sans]{dsfont}
\usepackage[round]{natbib}
\usepackage{bm}																									%matrix symbol
\usepackage{setspace}																						%Fu�noten (allgm.
\usepackage[colorlinks = true,
            linkcolor = blue,
            urlcolor  = blue,
            citecolor = blue,
            anchorcolor = blue]{hyperref}%Zeilenabst�nde)
\usepackage{threeparttable}
\usepackage{lscape}																							%Querformat
\usepackage[latin1]{inputenc}																		%Umlaute
\usepackage{graphicx}
\usepackage{amsmath}
\usepackage{amssymb}
\usepackage{fancybox}																						%Boxen und Rahmen
\usepackage{appendix}
\usepackage{listings}
\usepackage{xr}

\usepackage{enumerate}


%\usepackage{lineno}
%\linenumbers

        																%EURO Symbol
\usepackage{tabularx}
\usepackage{longtable,tabu}
\usepackage{subfig,float}																				%Mehrseitige Tabellen
\usepackage{color,colortbl}																			%Farbige Tabellen
\usepackage[left=2cm, right=2cm, top=2cm, bottom=2.5cm]{geometry} %Seitenr�nder
%\usepackage[normal]{caption2}[2002/08/03]												%Titel ohne float - Umgebung
\definecolor{lightgrey}{gray}{0.95}	%Farben mischen
\definecolor{grey}{gray}{0.85}
\definecolor{darkgrey}{gray}{0.80}

\newcommand{\mc}{\multicolumn}

\usepackage{tikz}
\usetikzlibrary{positioning}

\usepackage[labelfont=bf]{caption}
\captionsetup[table]{skip=10pt}

\usepackage{url}  % Used for linebreaks in verbatim statements

\newtheorem{Definition}{Definition}
\newtheorem{Remark}{Remark}
\newtheorem{Lemma}{Lemma}
\newtheorem{Theorem}{Theorem}
\newtheorem{Excercise}{Excercise}
\newtheorem{Result}{Result}
\newtheorem{Proposition}{Proposition}
\newtheorem{Prediction}{Prediction}
\newtheorem{Solution}{Solution}
\newtheorem{Problem}{Problem}

\setlength{\skip\footins}{1.0cm}
\deffootnote[1em]{1.1em}{0em}{\textsuperscript{\thefootnotemark}}
\renewcommand{\arraystretch}{1.05}

\DeclareMathOperator*{\argmin}{arg\,min}
\DeclareMathOperator*{\argmax}{arg\,max}

\makeatletter
\newenvironment{manquotation}[2][2em]
  {\setlength{\@tempdima}{#1}%
   \def\chapquote@author{#2}%
   \parshape 1 \@tempdima \dimexpr\textwidth-2\@tempdima\relax%
   \itshape}
  {\par\normalfont\hfill--\ \chapquote@author\hspace*{\@tempdima}\par\bigskip}
\makeatother

\newenvironment{boenumerate}
{\begin{enumerate}\renewcommand\labelenumi{\textbf{(\theenumi)}}}
{\end{enumerate}}

\parindent0pt
