%!TEX root = ../main.tex
\documentclass[a4paper,12pt,bold]{scrartcl}

\usepackage{color,colortbl}																			%Farbige Tabellen
\usepackage{xcolor}

\usepackage{apacite}
\usepackage{bibentry}

% Packages for OSE font
\usepackage[semibold, scale=0.89]{plex-mono}  % To match the x-height of the Charter serif font
\usepackage[semibold, scale=0.89]{plex-sans}[2018/12/31]

\usepackage{footnote,placeins}
\makesavenoteenv{tabular}
%\usepackage{enumitem}
\usepackage{footmisc}
\newcommand*\diff{\mathop{}\!\mathrm{d}}
\renewcommand{\baselinestretch}{1.3}\normalsize
\newcommand{\vect}[1]{\mathbf{#1}}
\newcommand{\thin}{\thinspace}
\newcommand{\thick}{\thickspace}
\newcommand{\N}{\mathcal{N}}	%Normal Distribution
\newcommand{\U}{\mathrm{U}}	%Uniform Distribution
\newcommand{\D}{\mathrm{D}}	%Dirichlet Distribution
\newcommand{\W}{\mathrm{W}}	%Wishart Distribution
\newcommand{\E}{\mathrm{E}}		%Expectation
\newcommand{\Ind}{\mathbb{I}\,}	%Indicator Function

\newcommand{\bs}{\boldsymbol}
\newcommand{\var}{\mathrm{var}\thin}
\newcommand{\plim}{\mathrm{plim}\thin}
\newcommand{\cov}{\mathrm{cov}\thin}
\newcommand\indep{\protect\mathpalette{\protect\independenT}{\perp}}
\def\independenT#1#2{\mathrel{\rlap{$#1#2$}\mkern5mu{#1#2}}}
\usepackage{bbm}
%\usepackage{endfloat}
\renewcommand{\vec}[1]{\mathbf{#1}}

\usepackage{algpseudocode,tabularx,ragged2e}
\newcolumntype{C}{>{\centering\arraybackslash}X} % centered "X" column
\newcolumntype{L}{>{\arraybackslash}X} % centered "X" column

\usepackage{algorithmicx}

\usepackage{algorithm}

\let\Algorithm\algorithm
\renewcommand\algorithm[1][]{\Algorithm[#1]\setstretch{1.5}}

% OSE colors
\definecolor{OSEBlue}	{RGB}{  0,  76, 141}
\definecolor{OSEBlueLight}	{RGB}{ 48, 188, 237}
% General colors
\definecolor{lightgrey}{gray}{0.90}	%Farben mischen
\definecolor{grey}{gray}{0.85}
\definecolor{darkgrey}{gray}{0.65}
\definecolor{lightblue}{rgb}{0.8,0.85,1}

\newcolumntype{g}{>{\columncolor{gray}}c}

\usepackage{booktabs}
\usepackage{epigraph}
\usepackage[sans]{dsfont}
\usepackage[round]{natbib}
\usepackage{bm}																									%matrix symbol
\usepackage{setspace}																						%Fu�noten (allgm.
\usepackage[colorlinks = true,
            linkcolor = OSEBlue,
            urlcolor  = OSEBlue,
            citecolor = OSEBlue,
            anchorcolor = blue]{hyperref}%Zeilenabst�nde)
\usepackage{threeparttable}
\usepackage{lscape}																							%Querformat
\usepackage[latin1]{inputenc}																		%Umlaute
\usepackage{graphicx}
\usepackage{amsmath}
\usepackage{amssymb}
\usepackage{fancybox}																						%Boxen und Rahmen
\usepackage{appendix}
\usepackage{listings}
\usepackage{xr}
\usepackage{listings}
\usepackage{enumerate}


%\usepackage{lineno}
%\linenumbers
\graphicspath{{../material/}}
        																%EURO Symbol
\usepackage{tabularx}
\usepackage{longtable,tabu}
\usepackage{subfig,float}																				%Mehrseitige Tabellen
\usepackage{color,colortbl}																			%Farbige Tabellen
\usepackage[left=2cm, right=2cm, top=2cm, bottom=2.5cm]{geometry} %Seitenr�nder
%\usepackage[normal]{caption2}[2002/08/03]												%Titel ohne float - Umgebung
\definecolor{lightgrey}{gray}{0.95}	%Farben mischen
\definecolor{grey}{gray}{0.85}
\definecolor{darkgrey}{gray}{0.80}

\newcommand{\mc}{\multicolumn}

\usepackage{tikz}
\usetikzlibrary{positioning}

\usepackage[labelfont=bf]{caption}
\captionsetup[table]{skip=10pt}

\usepackage{url}  % Used for linebreaks in verbatim statements

\newtheorem{Definition}{Definition}
\newtheorem{Remark}{Remark}
\newtheorem{Lemma}{Lemma}
\newtheorem{Theorem}{Theorem}
\newtheorem{Excercise}{Excercise}
\newtheorem{Result}{Result}
\newtheorem{Proposition}{Proposition}
\newtheorem{Prediction}{Prediction}
\newtheorem{Solution}{Solution}
\newtheorem{Problem}{Problem}

\setlength{\skip\footins}{1.0cm}
\deffootnote[1em]{1.1em}{0em}{\textsuperscript{\thefootnotemark}}
\renewcommand{\arraystretch}{1.05}

\DeclareMathOperator*{\argmin}{arg\,min}
\DeclareMathOperator*{\argmax}{arg\,max}

\makeatletter
\newenvironment{manquotation}[2][2em]
  {\setlength{\@tempdima}{#1}%
   \def\chapquote@author{#2}%
   \parshape 1 \@tempdima \dimexpr\textwidth-2\@tempdima\relax%
   \itshape}
  {\par\normalfont\hfill--\ \chapquote@author\hspace*{\@tempdima}\par\bigskip}
\makeatother

% OSE font settings
% !TeX TXS-program:compile = txs:///pdflatex/
% !TeX TS-program = pdflatex
% !TeX TXS-program:bibliography = txs:///biber
% !BIB program = biber




%%%%%%%%%%%%%%%%%%%%%%%%%%%%%%%%%%%%
%%  FONT SETTINGS AND TYPOGRAPHY  %%
%%%%%%%%%%%%%%%%%%%%%%%%%%%%%%%%%%%%


%% Use Fira Sans as the sans-serif font.
%% Load first because it otherwise doesn't work properly.
%% ==>
%\usepackage[lining, scale=0.88]{FiraMono}  % To match the x-height of the Charter serif font
%\usepackage[book, semibold, lining, tabular, scale=0.88]{FiraSans}[2018/05/23]
%\makeatletter
%\@ifpackagelater{FiraSans}{2018/05/23}
%	{}
%	{\PackageError{FiraSans}
%	     {Outdated 'FiraSans' package}
%	     {Upgrade to FiraSans 2018/05/23 or newer!\MessageBreak
%	      Otherwise the sans-serif font may not work properly, or the compilation might fail.\MessageBreak
%	      This is a fatal error. I'm aborting now.}%
%	\endinput%
%   }
%\makeatother
%% In case you wish to use the ``light'' options of FiraSans: -->
%\ifthenelse{\isundefined{\AfterPackage}}{\usepackage{afterpackage}}{}
%\AfterPackage{siunitx}{%
%  \providecommand*{\lseries}{\fontseries{l}\selectfont}
%}
%% See https://tex.stackexchange.com/questions/164791/combination-of-siunitx-and-light-weight-font-fails-to-compile.
%% <--
%% <==

% This template also works with the ``IBM Plex Sans'' and ``IBM Plex Mono'' fonts,
% https://ctan.org/pkg/plex ==>
\usepackage[semibold, scale=0.89]{plex-mono}  % To match the x-height of the Charter serif font
\usepackage[semibold, scale=0.89]{plex-sans}[2018/12/31]
	% Earlier versions of the ``plex-sans'' package do not include support for LGR Greek.
% <==

%% This template also works with the ``Alegreya'' and ``Alegreya Sans'' fonts,
%% https://ctan.org/pkg/alegreya ==>
%\usepackage[semibold, scale=0.89]{plex-mono}  % To match the x-height of the Charter serif font
%\usepackage[lining, tabular, scale=1.02]{AlegreyaSans}[2018/08/02]
%	% Earlier versions of the ``Alegreya'' package do not include support for LGR Greek.
%% <==

%% This template also works with the ``Google Noto'' fonts,
%% https://ctan.org/pkg/noto ==>
%\usepackage[scale=0.85]{noto-mono}
%\usepackage[lining, tabular, scale=0.85]{noto-sans}[2019/01/11]
%	% Earlier versions of the ``noto'' package do not include support for LGR Greek.
%% <==

% Save the font family and font series definitions for later use: ==>
\newcommand{\savesffamily}{\sfdefault}
\makeatletter
\newcommand{\savesfmdseries}{\mdseries@sf}
\newcommand{\savesfbfseries}{\bfseries@sf}
\makeatother
% <==

% Use Bitstream Charter as the math font
% ==>
\usepackage[charter, greeklowercase=italicized, greekuppercase=italicized, sfscaled=false, ttscaled=false]{mathdesign}
%% Use Alegreya as the text font -->
%\usepackage[lining, tabular, scale=1.02]{Alegreya}[2018/08/02]
%\ifxetex \else \DisableLigatures[T]{family = {rm*}} \fi
% <--
%% Use Noto Serif as the text font -->
%\usepackage[lining, tabular, scale=0.85]{noto-serif}[2019/01/11]
%% <--
% Save the font family and font series definitions for later use:
% Use IBM Plex Serif as the text font -->
\usepackage[scale=0.89]{plex-serif}  %\usepackage[semibold, scale=0.89]{plex-serif}
% <--
\newcommand{\rmmdseries}{text}
\let \savermfamily   \rmdefault
\let \savermmdseries \rmmdseries  %\mddefault
\let \savermbfseries \bfdefault
% <==

% mathdesign provides upright Greek letters as \alphaup, \Betaup, etc. while other packages
% (e.g., unicode-math) call these \upalpha, \upBeta, etc. We make also the latter available.
% ==>
\makeatletter
\@for\@tempa:=%
	alpha,beta,gamma,delta,epsilon,zeta,eta,theta,iota,kappa,lambda,mu,nu,xi,%
	omicron,pi,rho,sigma,varsigma,tau,upsilon,phi,chi,psi,omega,digamma,%
	Alpha,Beta,Gamma,Delta,Epsilon,Zeta,Eta,Theta,Iota,Kappa,Lambda,Mu,Nu,Xi,%
	Omicron,Pi,Rho,Sigma,Tau,Upsilon,Phi,Chi,Psi,Omega,Digamma%
	\do{%
		\expandafter\let\csname up\@tempa\expandafter\endcsname\csname\@tempa up\endcsname%
	}%
\makeatother
% <==

% Provide a \nequiv symbol
\ifx\nequiv\undefined
	\newcommand{\nequiv}{\not\equiv}
\fi

% Use Bitstream Charter as the text font
% ==>
%\usepackage[scaled=.96, lining, sups]{XCharter}  % option ``sups'' not working properly with subimport
% Save the font family and font series definitions for later use:
\let \savermdefaultfortext \rmdefault
\let \savemddefaultfortext \mddefault
\let \savebfdefaultfortext \bfdefault
\renewcommand{\textellipsis}{\mbox{.{\kern.09em}.{\kern.09em}.}}
\makeatletter
\newcommand{\@makefnmarkorig}{%
	\hbox{\sufigures\hspace*{.04em}\@thefnmark\hspace*{.04em}}%
}  % Copied from XCharter.sty
\makeatother
% <==

\ifxetex
	% Do nothing.
\else
	\DisableLigatures[f]{family = {rm*, tt*}}
	% Disable the f* ligatures for Charter because the font does not provide sufficient support.
	% Disable the f* ligatures for the ``typewriter'' font because they make no sense for a monospaced font.
	\SetExtraKerning[unit=space]
		{encoding=*, family=*, series=*, size={*, normalsize, footnotesize}, font = */*/*/*/*}
		{\textemdash = {325, 325},
		           / = {100, 100},
		           : = { 50,   0},
		           ; = { 50,   0}}
\fi

\usepackage{xfrac}	% Provides \sfrac

% Since math mode uses a different font encoding, issuing \euro/\texteuro in math mode
% produces an incorrect sign. We fix this. ==>
\AtBeginDocument{%
	\renewcommand*{\euro}[1]{%
		\relax\ifmmode\text{\texteuro}#1\else\texteuro #1\fi%
	}%
}
% <==


\newenvironment{boenumerate}
{\begin{enumerate}\renewcommand\labelenumi{\textbf{(\theenumi)}}}
{\end{enumerate}}
