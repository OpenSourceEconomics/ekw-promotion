%!TEX root = ../main.tex
%-------------------------------------------------------------------------------
\section{Extensions}\label{Extensions}
%-------------------------------------------------------------------------------
We briefly discuss selected extensions to the baseline model that are our active areas of research:

\subsection{Uncertainty quantification}

\subsection{Robust decision-making} Individuals face ubiquitous uncertainties when faced with important decisions. Policy makers vote for climate change mitigation efforts facing uncertainty about future costs and benefits \citep{Barnett.2019}, while doctors decide on the timing of an organ transplant
in light of uncertainty about future patient health \citep{Kaufman.2017}. Economic models formalize the objectives, trade-offs, and uncertainties for such decisions. In these models, the treatment of uncertainty is often limited to risk as the model induces a unique probability distribution over sequences of possible futures. There is no role for ambiguity about the true model \citep{Knight.1921,Arrow.1951} and thus no fear of model misspecification. However, limits to knowledge lead to considerable ambiguity about how the future unfolds \citep{Hayek.1975,Hansen.2015}.\\

\noindent This creates the need for robust decision rules that work well over a whole range of different models, instead of a decision rule that is optimal for one particular model. Optimal decision rules are designed without any fear of misspecification, using a single model to inform decisions. They thus perform very well if that model turns out to be true. However, their performance is very sensitive and deteriorates rapidly in light of model misspecification. Robust decision rules explicitly account for such a possibility and their performance is less affected.\\

\noindent Methods from distributionally robust optimization \citep{Ben-Tal.2009,Wiesemann.2014,Rahimian.2019} and robust Markov decision processes \citep{Iyengar.2005,Nilim.2005,Wiesemann.2014} allow to construct decision rules that explicitly take potential model misspecification into account.

\subsection{Model validation}
