%!TEX root = ../main.tex
%-------------------------------------------------------------------------------
\subsection{Calibration procedure}\label{Estimation}
%-------------------------------------------------------------------------------
EKW models are calibrated to data on observed individual decisions and experiences to obtain information on reward functions, preference parameters, and transition probabilities \citep{Davidson.2003,Gourieroux.1996}. Given this information, the quantitative importance of competing economic mechanisms can be assessed and the effects of alternative public policies forecasted. This requires the parameterization of all elements of the model which we collect in $\theta$.\\

\noindent The econometrician has access to observations for $i = 1, \hdots, N$ individuals in each time period $t$. For every observation $(i, t)$ in the data, the researcher observes action $a_{it}$ and a subset $x_{it}$ of the state $s_{it}$. Therefore, from an researcher's point of view, we need to distinguish between two types of state variables $s_{it} = (x_{it}, \epsilon_{it})$. Variables $x_{it}$ are observed by the econometrician and the individual $i$ at time $t$, while $\epsilon_{it}$ are only observed by the individual. In addition, some realizations of the rewards $r_{it} = r(x_{it}, \epsilon_{it}, a_{it})$ are observed as well. In summary, the data $\mathcal{D}$ contains:
%
\begin{align*}
  \mathcal{D} = \{a_{it}, x_{it}, r_{it}: i = 1,2, \hdots, N; t = 1, \hdots, T_i\},
\end{align*}
where $T_i$ is the number of observations over which we observe individual $i$.\\

\noindent Different calibration procedures exist that address particularities of the available data. We briefly outline likelihood-based and simulation-based calibration  Whatever the calibration criterion, in order to evaluate it for every candidate parameterization of the model $\theta$ it is necessary to construct the optimal policy $\pi^*$. Therefore at each trial value of $\theta$, we need to solved the model using the backward induction procedure outlined in Algorithm \ref{Backward induction procedure}.

\paragraph{Likelihood-based} The individual chooses the alternative with the highest total value $a_t^{\pi^*}(s_t)$ which is determined by the complete state $s_t$. However, researchers only observe the subset $x_t$. Given parametric assumptions about the distribution of $\epsilon$, we can determine the probability $p_{it}(a_{it}, r_{it} \mid x_{it}, \theta)$ of individual $i$ at time $t$ choosing $a_{it}$ and receiving $r_{it}$ given $x_{it}$.\\

\noindent The likelihood function $\mathcal{L}(\theta\mid\mathcal{D})$ captures the probability of the observed data as a function of $\theta$ and the goal of likelihood-based estimation is to find the value of the model parameters $\theta$ that maximizes it
%
\begin{align*}
  \hat{\theta} \equiv \argmax_{\theta \in \Theta}  \underbrace{\prod^N_{i= 1} \prod^{T_i}_{t= 1}\, p_{it}(a_{it}, r_{it} \mid x_{it}, \theta)}_{\mathcal{L}(\theta\mid\mathcal{D})}
\end{align*}

\paragraph{Simulation-based}

\begin{itemize}
  \item This issue is under active investigation in our group and this section will be fleshed out further by Annica Gehlen as our work progresses.
\end{itemize}
