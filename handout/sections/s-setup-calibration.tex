%!TEX root = ../main.tex
%-------------------------------------------------------------------------------
\subsection{Calibration procedure}\label{Estimation}
%-------------------------------------------------------------------------------
EKW models are calibrated to data to obtain information on preference parameters and transition probabilities \citep{Davidson.2003,Gourieroux.1996}. Given this information, the quantitative importance of competing economic mechanisms can be assessed and the effects of public policies predicted. This requires the parameterization of all elements of the model which we collect in $\theta$.

\paragraph{Data} The econometrician has access to panel data for $N$ individuals. For every observation $(i, t)$ in the panel data set, the researcher observes action $a_{it}$ and a subvector $x_{it}$ of the state vector. Therefore, from an econometricians point of view, we need to distinguish between two types of state variables $s_{it} = (x_{it}, \epsilon_{it})$. Variables $x_{it}$ that are observed by the econometrician and the individual $i$ at time $t$ and those that are only observed by the individual $\epsilon_{it}$.  In addition, also some realizations of the rewards $r_{it} = r(x_{it}, \epsilon_{it}, a_{it})$.
%
\begin{align*}
  \mathcal{D} = \{a_{it}, x_{it}, r_{it}: i = 1,2, \hdots, N; t = 1, \hdots, T_i\},
\end{align*}
where $T_i$ is the number of observations over which we observe individual $i$.

\paragraph{Procedures} We briefly outline maximum likelihood estimation and the method of simulated moments.  Whatever the estimation criterion, in order to evaluate it for a particular value of $\theta$  it is necessary to construct the optimal policy $\pi^*$. Therefore at each trial value of $\theta$ the whole model needs to solved by the backward induction algorithm.

\paragraph{Likelihood-based} The individual chooses the alternative with the highest total value $a_t^{\pi^*}(s_t)$ which is determined by the complete state and rewards are also determined by $s$. However, the econometrician only observes the subset $x$. Given parametric assumptions about the distribution of $\epsilon$, we can determine the probability $p_{it}(a_{it}, r_{it} \mid x_{it}, \theta)$ of individual $i$ at time $t$ choosing $d_{it}$ and receiving $r_{it}$ given $x_{it}$.
%
\begin{align*}
  \mathcal{L}(\theta\mid\mathcal{D}) \equiv \prod^N_{i= 1} \prod^{T_i}_{t= 1}\, p_{it}(a_{it}, r_{it} \mid x_{it}, \theta)
\end{align*}

\noindent The goal of likelihood-based estimation is to find the value of the model parameters $\theta$ that maximize the likelihood function:
%
\begin{align*}
\hat{\theta} \equiv \argmax_{\theta \in \Theta} \mathcal{L}(\theta\mid\mathcal{D})
\end{align*}

\paragraph{Simulation-based} $\hdots$ to be written
