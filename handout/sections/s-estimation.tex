%!TEX root = ../main.tex
%-------------------------------------------------------------------------------
\section{Estimation procedures}\label{Estimation}
%-------------------------------------------------------------------------------
\paragraph{Available information} The econometrician has access to panel data for $N$ individuals. For every observation $(i, t)$ in the panel data set, the researcher observes action $a_{it}$ and a subvector $x_{it}$ of the state vector. Therefore, from an econometricians point of view, it is useful to distinguish between two types of state variables $s = (x, \epsilon)$. Variables $x$ that are observed by the econometrician and the individual and those that are only observed by the individual $\epsilon$.  In addition, also some realizations of the rewards $r_{it} = r(x_{it}, \epsilon_{it}, a_{it})$.
%
\begin{align*}
  \mathcal{D} = \{a_{it}, x_{it}, r_{it}: i = 1,2, \hdots, N; t = 1, \hdots, T_i\}
  \},
\end{align*}
where $T_i$ is the number of observations over which we observe individual $i$.

\paragraph{Parameterization} The models need to be parameterized functional forms and the distribution of unobservables. EKW models are calibrated to obtain information on structural parameters of preferences and the transition probabilities that reproduce key economic patterns of interest in observed sources such as administrative data sets and the like. This allows to assess the quantitative importance of competing economic mechanisms and then predict the effects of public policies. We collect all parameters of the model in $\theta$.

\paragraph{Procedures} We briefly outline maximum likelihood estimation \citep{Fisher.1922} and the method of simulated moments \citep{McFadden.1989}.  Whatever the estimation criterion, in order to evaluate it for a particular value of $\theta$  it is necessary to know the optimal decision rules $d^{\pi^*}$. Therefore at each trial value of $\theta$ the dynamic programming model needs to be solved. More detailed information is available in the excellent textbooks by \citet{Davidson.2003} and \citet{Gourieroux.1996}.

\paragraph{Likelihood-based estimation} The individual chooses the alternative with the highest total value $d^*(s)$ which is determined by the complete state space and rewards are also determined by $s$. However, the econometrician only observes the subset $x$ and thus can only determine the probability $p_{it}(a_{it}, r_{it} \mid x_{it}, \theta)$ of individual $i$ at time $t$ choosing $d_{it}$ and receiving $r_{it}$ given $x_{it}$.
%
\begin{align*}
  \mathcal{L}(\theta\mid\mathcal{D}) = \prod^N_{i= 1} \prod^T_{t= 1} p_{it}(a_{it}, r_{it} \mid x_{it}, \theta)
\end{align*}

The goal of likelihood-based estimation is to find the value of the model parameters $\theta$ that maximize the likelihood function:
%
\begin{align*}
\hat{\theta} = \argmax_{\theta \in \Theta} \mathcal{L}(\theta\mid\mathcal{D})
\end{align*}

\paragraph{Simulation-based estimation} ...


The goal of simulation-based estimation
