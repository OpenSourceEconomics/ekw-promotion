% !TeX TXS-program:compile = txs:///pdflatex/
% !TeX TS-program = pdflatex
% !TeX TXS-program:bibliography = txs:///biber
% !BIB program = biber




%%%%%%%%%%%%%%%%%%%%%%%%%%%%%%%%%%%%%%%%%
%%  FUNDAMENTAL PACKAGES AND COMMANDS  %%
%%%%%%%%%%%%%%%%%%%%%%%%%%%%%%%%%%%%%%%%%


\usepackage{ifxetex}  % Detect if engine is XeTeX/XeLaTeX
\usepackage{ifthen}

\usepackage[utf8]{inputenc}  % so that umlauts can be input without having to use TeX code

\ifxetex
	\usepackage[euenc]{fontspec}
\else
	\usepackage[LGR, T1]{fontenc}  % LGR needed for sansserif math
\fi

\usepackage[ngerman, american, USenglish]{babel}  % German and US English hyphenation and quotation marks
\selectlanguage{USenglish}
\usepackage[ngerman, USenglish]{isodate}
\usepackage[babel, german=quotes]{csquotes}  % Needed for correct German quotes via BibLaTeX's \mkbibquote{...}
%\usepackage[babel, german=guillemets]{csquotes}  % Needed for correct German quotes via BibLaTeX's \mkbibquote{...}

\usepackage{calc}  % Enables calculations for lengths; provides, e.g., \widthof{text}
\usepackage{fp}  % Enables calculations in LaTeX
% \usepackage[nomessages]{fp}

\usepackage{etoolbox}  % Enables manipulating LaTeX commmands via \preto, \appto, \patchcmd, etc.
\usepackage{xpatch}  % Enables manipulating LaTeX commmands via \xpatchcmd etc.
\usepackage{letltxmacro}
\usepackage{xparse}

%\usepackage{geometry}  % See geometry.pdf to learn the layout options. There are lots.
\usepackage[mathscr]{eucal}
%\geometry{a4paper, top=20mm, bottom=20mm, right=20mm, left=30mm}
%\geometry{landscape}  % Activate for for rotated page geometry

\usepackage{ragged2e}
% Provides, among others, the \RaggedRight environment, which is \raggedright but with hyphenation enabled.
%\renewcommand{\raggedright}{\RaggedRight}

\usepackage{import}	 % To allow for relative paths in nested \input's (\import's)

% \usepackage{float} % damit die figure da ist wo sie sein soll
\usepackage{placeins}  % Improve placing of floats (figures, tables), provides \FloatBarrier

% DO NOT USE \usepackage{amssymb}!
% The AMS symbols are included in the mathdesign package or the fourier package.
\usepackage{amsmath}
\MakeRobust{\eqref}
	% See https://tex.stackexchange.com/questions/61764/eqref-in-captions-with-mathtools
\renewcommand{\eqref}[1]{(\ref{#1})}  % Necessary to make the font switch from serif to sansserif where required.
\usepackage{amsthm}	% provides \newtheoremstyle
\usepackage{mathtools}
%\mathtoolsset{centercolon}
	% This makes the compilation fail in combination with tikz. See
	% https://tex.stackexchange.com/questions/89467/why-does-pdftex-hang-on-this-file.
% Inspired by https://tex.stackexchange.com/questions/251460/how-to-put-symbols-of-equal-size-on-top-of-each-other
\newcommand{\succeqq}{%
  \mathrel{%
    \vcenter{\offinterlineskip
      \ialign{##\cr$\succ$\cr\noalign{\kern 1pt}$=$\cr}%
    }%
  }%
}
\newcommand{\nsucceqq}{\mathrel{\not\succeqq}}
\usepackage[
	colorlinks=true,
	linkcolor=black,
	citecolor=black,
	filecolor=black,
	urlcolor=black,
	bookmarks=true,
	bookmarksnumbered=true,
	bookmarksopenlevel=2,
	pdfstartview=Fit,
	pdfpagelayout=SinglePage,
	plainpages=false,
	pdfpagelabels=true
]{hyperref}
\urlstyle{same}   % Sets URLs in the text font instead of the typewriter font
\Urlmuskip = 0mu\relax  % Prevent additional whitespace before/after breakable characters in URLs
%\setlength{\Urlmuskip}{0mu}
\newcommand{\email}[1]{\href{mailto:#1}{\nolinkurl{#1}}}
% Change ``([sub]sub)section'' to ``Section'' in \autoref
% and add na
% ==>
\addto\extrasUSenglish{%
	\renewcommand{\chapterautorefname}{Chapter}%  instead of ``chapter''
	\renewcommand{\sectionautorefname}{Section}%  instead of ``section''
	\renewcommand{\subsectionautorefname}{Section}%  instead of ``subsection''
	\renewcommand{\subsubsectionautorefname}{Section}%  instead of ``subsubsection''
}
\addto\extrasamerican{%
	\renewcommand{\chapterautorefname}{Chapter}%  instead of ``chapter''
	\renewcommand{\sectionautorefname}{Section}%  instead of ``section''
	\renewcommand{\subsectionautorefname}{Section}%  instead of ``subsection''
	\renewcommand{\subsubsectionautorefname}{Section}%  instead of ``subsubsection''
}
\addto\extrasngerman{%
	\renewcommand{\subsectionautorefname}{Abschnitt}%  instead of ``Unterabschnitt''
	\renewcommand{\subsubsectionautorefname}{Abschnitt}%  instead of ``Unterunterabschnitt''
}
\newcommand*{\Appendixautorefname}{Appendix}
	% See https://tex.stackexchange.com/questions/207744/no-autoref-name-for-appendix
\newcommand*{\hypothesisautorefname}{Hypothesis}
\newcommand*{\resultautorefname}{Result}
% <==
% Enclose the back references in the bibliography to the pages on which a reference is cited in square brackets (Econometrica style):
% (only applicable if using BibTeX instead of BibLaTeX)
%\let \backrefold \backref
%\renewcommand*{\backref}[1]{[\backrefold{#1}]}

% \usepackage{cleveref}	% Provides \cref{...} etc. for flexible referencing.

%% Make \autoref include parentheses around equation nunmbers -->
%% See https://tex.stackexchange.com/questions/191376/autoref-and-braces-around-equation-number
%\makeatletter
%\let\oldtheequation\theequation
%\renewcommand\tagform@[1]{\maketag@@@{\ignorespaces#1\unskip\@@italiccorr}}
%\renewcommand\theequation{(\oldtheequation)}
%\makeatother
%% <--

\usepackage{verbatim}
\let\ignore=\comment
\let\endignore=\endcomment

%% Command to suppress text:
%% See https://tex.stackexchange.com/questions/97347/selectively-suppress-generation-of-typeset-output
%% ==>
%\makeatletter
%\font\dummyft@=dummy \relax
%\def\suppress{%
%	\begingroup\par
%	\parskip\z@
%	\offinterlineskip
%	\baselineskip=\z@skip
%	\lineskip=\z@skip
%	\lineskiplimit=\maxdimen
%	\dummyft@
%	\count@\sixt@@n
%	\loop\ifnum\count@ >\z@
%	\advance\count@\m@ne
%	\textfont\count@\dummyft@
%	\scriptfont\count@\dummyft@
%	\scriptscriptfont\count@\dummyft@
%	\repeat
%	\let\selectfont\relax
%	\let\mathversion\@gobble
%	\let\getanddefine@fonts\@gobbletwo
%	\tracinglostchars\z@
%	\frenchspacing
%	\hbadness\@M}
%\def\endsuppress{\par\endgroup}
%\makeatother
%% <==




%%%%%%%%%%%%%%%%%%%%%%%%%%%%%%%%%
%%  GRAPHICS-RELATED PACKAGES  %%
%%%%%%%%%%%%%%%%%%%%%%%%%%%%%%%%%


\usepackage{graphicx}
\DeclareGraphicsRule{.tif}{png}{.png}{`convert #1 `dirname #1`/`basename #1 .tif`.png}

\usepackage{epstopdf}  % Has to be loaded after graphic{s,x}

\usepackage[table]{xcolor}
\definecolor{UBonnBlue}   {RGB}{0007,0082,0154}
\definecolor{darkblue}    {rgb}{0.00,0.20,0.40}
\definecolor{darkred}     {rgb}{0.80,0.00,0.00}
\colorlet   {darkred25}   {darkred!25!white}
\definecolor{customgreen} {rgb}{0.15,0.55,0.00}
\definecolor{custompurple}{rgb}{0.15,0.00,0.75}

\usepackage{pgf, pgfarrows, pgfnodes, pgfshade}
\usepackage{pgfplots}

\usepackage{tikz}
\usetikzlibrary{mindmap, trees, patterns}

\usepackage{pdflscape}
% To set single pages in landscape ortientation

\usepackage{afterpage}
% To wrap text around landscape-oriented pages




%%%%%%%%%%%%%%%%%%%%%%%%%%%%%%%%
%%  ADVANCED TEXT FORMATTING  %%
%%%%%%%%%%%%%%%%%%%%%%%%%%%%%%%%


\usepackage[full]{textcomp}  % ``full'' option tequired some packages (e.g., ``newtxtext'')
\usepackage{xfrac}	% Provides \sfrac; loads textcomp (without ``full'' option)

\usepackage{soul}  % Provides a~highlighting command, \hl{...}, and a \caps{...} command

% Allow for fine-grained scaling of font sizes
% ==>
\usepackage{relsize}
\renewcommand\RSpercentTolerance{1}
% Enabling slightly reduced font for CAPS:
\ifxetex
	\renewcommand{\caps}[1]{\textscale{0.96}{\addfontfeature{LetterSpace=5}\MakeUppercase{#1}}}
\else
	\renewcommand{\caps}[1]{\textscale{0.96}{\textls[35]{\MakeUppercase{#1}}}}
\fi
% <==

\frenchspacing	% Prevent excessively large whitespace after periods
\sloppy




%%%%%%%%%%%%%%%%%%%%%%%%%%%%%%%%%%%%
%%  COMMANDS FOR TROUBLESHOOTING  %%
%%%%%%%%%%%%%%%%%%%%%%%%%%%%%%%%%%%%


\usepackage{printlen}  % Enables outputting the current values of lengths

\usepackage[math]{blindtext}
\makeatletter
\def\blindtext@american{}
\renewcommand{\blindmathpaper}{%
	\blindtext
	\blindtext@formula\par
	\blindtext
	\blindtext@formula
	\blindtext
	\blindtext@formula\par
	\blindtext
	\blindtext@formula
	\blindtext
	\blindtext@formula\par
	\blindtext\relax%
}
\makeatother
\setcounter{blindtext}{1}
\setcounter{Blindtext}{1}
% The ``blindtext'' package does not recognize ``USenglish'' as identical to ``american''.
% Fix this -->
\LetLtxMacro{\blindtextblindtext}{\blindtext}
\LetLtxMacro{\blindtextBlindlist}{\Blindlist}
\LetLtxMacro{\blindtextBlindtext}{\Blindtext}
\RenewDocumentCommand{\blindtext}{O{\value{blindtext}}}{%
	\begingroup%
	\iflanguage{USenglish}{\selectlanguage{american}}{}%
	\blindtextblindtext[#1]%
	\endgroup%
}
\RenewDocumentCommand{\Blindtext}{O{\value{blindtext}} O{\value{Blindtext}}}{%
	\begingroup%
	\iflanguage{USenglish}{\selectlanguage{american}}{}%
	\blindtextBlindtext[#1][#2]%
	\endgroup%
}
\RenewDocumentCommand{\Blindlist}{m O{\value{blindlist}}}{%
	\begingroup%
	\iflanguage{USenglish}{\selectlanguage{american}}{}%
	\blindtextBlindlist{#1}[#2]%
	\endgroup%
}
% Based on https://tex.stackexchange.com/questions/299954/styling-blindtext-and-blindtext-aka-renewcommand-with-optional-arguments
% <==

%\usepackage{showframe}
%\renewcommand*\ShowFrameColor{\color{magenta}}
%\usepackage[pagewise]{lineno}
%\addtolength{\linenumbersep}{7.5pt}
%\renewcommand{\linenumberfont}{\sffamily\tiny\color{gray}}
%\linenumbers
% An auxiliary command to display the current font settings -->
\makeatletter
\newcommand{\showfont}{{%
	\color{magenta}
	\textit{Encoding:} \f@encoding{},
	\textit{family:}   \f@family{},
	\textit{series:}   \f@series{},
	\textit{shape:}    \f@shape{},
	\textit{size:}     \f@size{}.
}}
\makeatother
\newcommand{\showfamily}{\f@family{}}
% <--

\makeatletter
\newcommand*{\checkgreekletters}{%
	\@for\@tempa:=%
	alpha,beta,gamma,delta,epsilon,zeta,eta,theta,iota,kappa,lambda,mu,nu,xi,%
	omicron,pi,rho,sigma,varsigma,tau,upsilon,phi,chi,psi,omega,digamma,%
	Alpha,Beta,Gamma,Delta,Epsilon,Zeta,Eta,Theta,Iota,Kappa,Lambda,Mu,Nu,Xi,%
	Omicron,Pi,Rho,Sigma,Tau,Upsilon,Phi,Chi,Psi,Omega,Digamma%
	\do{$\csname\@tempa\endcsname,$ }%
}
\makeatother

\usepackage{fonttable}
% Color slot numbers in \xfonttable gray instead of black -->
\makeatletter
\renewcommand*\f@placedecimal[2]{#1\ {\color{gray}\tiny #2}}
\renewcommand*{\f@toct}[1]{\hbox{\color{gray}\rmfamily\'{}\kern-.2em\itshape#1\/\kern.05em}} % octal constant
\renewcommand*{\f@thex}[1]{\hbox{\color{gray}\rmfamily\H{}\ttfamily#1}} % hexadecimal constant
\makeatother
% <--
