
\makeatletter
\newcommand{\Tw}{\strip@pt\textwidth}
\makeatother
\titlegraphic{%
	\makebox(\Tw,0)[rt]{%
		\makebox(\Tw,185.5)[rb]{%
			\includegraphics[height=1.33cm]{ose-logo-rgb.pdf}\:\:%
		}%
	}%
	\vspace{-12pt}%
}

\newcommand{\backupbegin}{
   \newcounter{framenumbervorappendix}
   \setcounter{framenumbervorappendix}{\value{framenumber}}
}
\newcommand{\backupend}{
   \addtocounter{framenumbervorappendix}{-\value{framenumber}}
   \addtocounter{framenumber}{\value{framenumbervorappendix}}
}


\usepackage{subfloat}
\usepackage{subfig}
\usepackage{multicol}
\usepackage{verbatim}

%\newcommand{\U}{\mathcal{U}}	%Uniform Distribution
\newcommand{\G}{\mathcal{G}}
% \newcommand{\dd}{\mathup{d}}  % Differential d
\usepackage{physics}  % Provides, among others, a \dd command for derivatives/differentials
\renewcommand{\diffd}{\mathup{d}}

%\newcommand{\explanation}[1]{%
%        \raisebox{0pt}[0pt][0pt]{%
%                \parbox[t]{0.75\textwidth}{%
%                        \textcolor{OSEBlue}{#1}%
%                }%
%        }%
%}

% Shape of \theta
% Inspired by https://tex.stackexchange.com/questions/251888/slant-character-manually-in-tex-coding -->
\newcommand{\slanted}[1]{{%
    \tikz[baseline=(fakeitalic.base), xslant=tan(11.25)]%
        \node[inner sep=0pt, xslant=tan(11.25)](fakeitalic){#1};%
}}
\renewcommand{\thetaLGR}{{%
    \mathord{\textup{\fontencoding{LGR}\selectfont{\kern-0.075em}\slanted{j}{\kern-0.075em}}}%
}}
\renewcommand{\vartheta}{{%
    \mathord{\textformath{\fontencoding{LGR}\selectfont j}}%
}}
% <--

\AtBeginSection[]{{%
	\setbeamercolor{background canvas}{bg=OSEBlue, fg=white}%
	\colorlet{SpotColor}{white}%
	\begin{frame}[standout, c]{~}
		%\vfill
		\usebeamerfont{title}\hypersetup{linkcolor=white}%
		\insertsection
		%\vfill
	\end{frame}%
	\colorlet{SpotColor}{OSEBlue}%
}}


%\newcommand{\E}{\mathrm{E}}		%Expectation
\newcommand{\Ind}{\,\mathbb{I}\,}	%Indicator Function
%\newcommand{\ind}{{\textbf{I}}}
\newcommand{\btheta}{\bm{\theta}}
\newcommand{\bTheta}{\bm{\Theta}}
%\newcommand{\N}{\mathcal{N}}		%Expectation

\usepackage{algorithmicx}
\usepackage{algpseudocode,tabularx,ragged2e}
\newcolumntype{C}{>{\centering\arraybackslash}X} % centered "X" column
\newcolumntype{L}{>{\arraybackslash}X} % centered "X" column
\newcommand{\M}{\mathcal{M}}

\usepackage{algorithmicx}

\usepackage{algorithm}

\let\Algorithm\algorithm
\renewcommand\algorithm[1][]{\Algorithm[#1]\setstretch{1.5}}

% OSE colors

%\newcommand{\argmin}{\arg\min}
%\newcommand{\argmax}{\arg\max}

\usepackage{tikz}
\usetikzlibrary{trees,shapes,arrows,decorations.pathmorphing,backgrounds,positioning,fit,petri}

\tikzset{forestyle/.style = {rectangle, thick, minimum width = 5cm, minimum height = 0.5cm, text width = 4.5cm, outer sep = 1mm},
	pre/.style={<-, shorten <=1pt, >=stealth, ultra thick},
	extend/.style={<-,dashed, shorten <=1pt, >=stealth, ultra thick}}
