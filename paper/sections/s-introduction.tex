%!TEX root = ../main.tex
%-------------------------------------------------------------------------------
\section{Introduction}
%-------------------------------------------------------------------------------
Economists use structural microeconometric models to study individual decision-making. These models specify the objective of individuals, their economic environment, and the institutional and informational constraints under which they operate. Calibration of the model to observed data on individual decisions and experiences allows quantifying the importance of competing economic mechanisms in determining economic outcomes and forecasting the effects of policy proposals \citep{Wolpin.2013}.\\

\noindent We restrict our exposition to the class of Eckstein-Keane-Wolpin (EKW) models \citep{Adda.2017, Blundell.2016, Keane.1997}. Labor economists use them to study human capital investment decisions. Human capital comprises the knowledge, skills, competencies, and attributes embodied in individuals facilitating the creation of personal, social, and economic well-being \citep{Becker.1964}. Differences in human capital attainment lead to inequality in various life outcomes such as labor market success and health across and within countries \citep{OECD.2001}. \\

\noindent In \citet{Bhuller.2018}, for example, we apply an EKW model to analyze the mechanisms determining schooling decisions in Norway. We calibrate the model using Norwegian population panel data with nearly career-long earnings histories. After validating our model using a mandatory schooling reform, we gain insights into the underlying economic mechanisms that generate the effects of the policy and forecast the impacts of several policy alternatives.\\

\noindent We offer this handout to facilitate transdisciplinary collaboration in the future development of EKW models. We first describe their economic framework, mathematical formulation, and calibration procedure. We then turn to the seminal model outlined in \citet{Keane.1997} as an example and present our group's ensemble of research codes that allow for its specification, simulation, and calibration. Finally, we summarize our efforts drawing on research outside economics to address the computational challenges in applying EKW models and improve the reliability and interpretability of their results.\\

\noindent Throughout, we only offer a limited number of seminal references and textbooks that invite further study. We introduce acronyms and symbols as needed, and our notation draws on the reviews by \cite{Aguirregabiria.2010}, \cite{Arcidiacono.2011}, and \cite{Puterman.1994}.
