% !TEX root = ../main.tex
%-------------------------------------------------------------------------------
\section{Computational implementation}\label{Computational implementation}
%-------------------------------------------------------------------------------
We use the same computational implementation as in \citet{Keane.1997}. We outline the per-period utility functions for each of the five alternatives. We first focus on their common overall structure and then present their parameterization. Throughout we provide the economic motivation for their specification.\\

\noindent We follow individuals over their working life from young adulthood at age 16 to retirement at age 65. The decision period $t = 16, \dots, 65$  is a school year and individuals decide $a\in\mathcal{A}$ whether to work in a blue-collar or white-collar occupation ($a = 1, 2$), to serve in the military $(a = 3)$, to attend school $(a = 4)$, or to stay at home $(a = 5)$.\\

\noindent Individuals are initially heterogeneous. They differ with respect to their initial level of completed schooling $h_{16}$ and have one of four different $\mathcal{J} = \{1, \hdots, 4\}$ alternative-specific skill endowments $\bm{e} = \left(e_{j,a}\right)_{\mathcal{J} \times \mathcal{A}}$.\\

\noindent The per-period utility $u_a(\cdot)$ of each alternative consists of a non-pecuniary utility $\zeta_a(\cdot)$ and, at least for the working alternatives, an additional wage component $w_a(\cdot)$. Both depend on the level of human capital as measured by their alternative-specific skill endowment $\bm{e}$, years of completed schooling $h_t$, and occupation-specific work experience $\bm{k_t} = \left(k_{a,t}\right)_{a\in\{1, 2, 3\}}$. The per-period utility functions are influenced by last-period choices $a_{t -1}$ and alternative-specific productivity shocks $\bm{\epsilon_t} = \left(\epsilon_{a,t}\right)_{a\in\mathcal{A}}$ as well. Their general form is given by:
%
\begin{align*}
u_a(\cdot) =
\begin{cases}
    \zeta_a(\bm{k_t}, h_t, t, a_{t -1})  + w_a(\bm{k_t}, h_t, t, a_{t -1}, e_{j, a}, \epsilon_{a,t})                & \text{if}\, a \in \{1, 2, 3\}  \\
    \zeta_a(\bm{k_t}, h_t, t, a_{t-1}, e_{j,a}, \epsilon_{a,t})                                                  &  \text{if}\, a \in \{4, 5\}.
\end{cases}
\end{align*}
%
Work experience $\bm{k_t}$  and years of completed schooling $h_t$ evolve deterministically.
%
\begin{align*}
k_{a,t+1} = k_{a,t} + \ind[a_t = a]  &\qquad \text{if}\, a \in \{1, 2, 3\} \\
h_{t + 1\phantom{,a}} = h_{t\phantom{,a}} +   \ind[a_t = 4]  &\qquad
\end{align*}
%
\noindent The productivity shocks are uncorrelated across time and follow a multivariate normal distribution with mean $\bm{0}$ and covariance matrix $\bm{\Sigma}$. Given the structure of the utility functions and the distribution of the shocks, the state at time $t$ is $s_t = \{\bm{k_t}, h_t, t, a_{t -1}, \bm{e},\bm{\epsilon_t}\}$.\\

\noindent Empirical and theoretical research from specialized disciplines within economics informs the exact specification of $u_a(\cdot)$. We now discuss each in detail.
%-------------------------------------------------------------------------------
\subsection{Per-period utility for working alternatives}
%-------------------------------------------------------------------------------
The per-period utility for working alternatives is the sum of a non-pecuniary utility and a wage component. The wage component $w_{a}(\cdot)$ is given by the product of the market-equilibrium rental price $r_{a}$ and an occupation-specific skill level $x_{a}(\cdot)$. The latter is determined by the overall level of human capital.
%
\begin{align*}
w_{a}(\cdot) = r_{a} \, x_{a}(\cdot)
\end{align*}
%
This specification leads to a standard logarithmic wage equation in which the constant term is the skill rental price $\ln(r_{a})$.\\

\noindent The occupation-specific skill level $x_{a}(\cdot)$ is determined by a skill production function, which includes a deterministic component $\Gamma_a(\cdot)$ and a multiplicative stochastic productivity shock $\epsilon_{a,t}$.
%
\begin{align}
    x_{a}(\bm{k_t}, h_t, t, a_{t-1}, e_{j, a}, \epsilon_{a,t}) & = \exp \big( \Gamma_{a}(\bm{k_t},  h_t, t, a_{t-1}, e_{j,a}) \cdot \epsilon_{a,t} \big) \nonumber
\end{align}
%-------------------------------------------------------------------------------
\subsubsection*{White-collar}
%-------------------------------------------------------------------------------
Equation (\ref{eq:SkillLevelWhiteCollar}) shows the parameterization of the deterministic component of the skill production function.
%
\begin{align}\label{eq:SkillLevelWhiteCollar}
    \Gamma_1(\bm{k_t}, h_t, t, a_{t-1}, e_{j, 1}) = e_{j,1} & + \beta_{1,1} \cdot h_t + \beta_{1, 2} \cdot \ind[h_t \geq 12] + \beta_{1,3} \cdot \ind[h_t\geq 16]\\ \nonumber
                                  & + \gamma_{1, 1} \cdot  k_{1,t} + \gamma_{1,2} \cdot  (k_{1,t})^2 + \gamma_{1,3} \cdot  \ind[k_{1,t} > 0] \\ \nonumber
                                & + \gamma_{1,4} \cdot  t + \gamma_{1,5} \cdot \ind[t < 18]\\ \nonumber
                                  & + \gamma_{1,6} \cdot \ind[a_{t-1} = 1] + \gamma_{1,7} \cdot  k_{2,t} + \gamma_{1,8} \cdot  k_{3,t} \nonumber
\end{align}
%
\noindent Part of the skill production function is motivated by \citet{Mincer.1958, Mincer.1974} and hence linear in years of completed schooling $\beta_{1,1}$, quadratic in experience ($\gamma_{1,1}, \gamma_{1,2}$), and separable between the two of them. The skill production function respects sheep-skin effects ($\beta_{1,2}, \beta_{1,3}$) and the gain of having worked in the same occupation at least once before $\gamma_{1,3}$ \citep{Spence.1973, Jaeger.1996}. Skill production depends linearly on the age $\gamma_{1,4}$ and on being a minor $\gamma_{1,5}$. Work experience depreciates when working in a different occupation ($\gamma_{1,6}$) but is potentially transferable across occupations ($\gamma_{1,7}, \gamma_{1,8}$). \\


\noindent Equation (\ref{eq:NonWageWhiteCollar}) shows the parameterization of the non-pecuniary utility from working in a white-collar occupation.
%
\begin{align}\label{eq:NonWageWhiteCollar}
\zeta_{1}(\bm{k_t}, h_t, a_{t-1})  = \alpha_1  &+ c_{1,1} \cdot \ind[a_{t-1} \neq 1] + c_{1,2} \cdot \ind[k_{1,t} = 0] \\ \nonumber
                            & + \vartheta_1 \cdot \ind[h_t \geq 12] + \vartheta_2 \cdot \ind[h_t \geq 16] + \vartheta_3 \cdot \ind[k_{3,t} = 1]
\end{align}
%
A constant $\alpha_1$ captures the net monetary-equivalent of on the job amenities. The non-pecuniary utility includes mobility and search costs $c_{1,1}$, which are higher for individuals who never worked in a white-collar occupation before $c_{1,2}$. The non-pecuniary utilities capture returns from a high school $\vartheta_1$ and a college $\vartheta_2$ degree. Additionally, there is a detrimental effect of leaving the military early after one year $\vartheta_3$.\\
%-------------------------------------------------------------------------------
\subsubsection*{Blue-collar}
%-------------------------------------------------------------------------------
The per-period utility from working in a blue-collar occupation is specified analogously. Equation (\ref{eq:SkillLevelBlueCollar}) shows the parameterization of the deterministic component of the skill production function.
%
\begin{align}\label{eq:SkillLevelBlueCollar}
    \Gamma_2(\bm{k_t}, h_t, t, a_{t-1}, e_{j,2}) = e_{j,2} & + \beta_{2,1} \cdot h_t + \beta_{2, 2} \cdot \ind[h_t \geq 12] + \beta_{2,3} \cdot \ind[h_t\geq 16] \\\nonumber
    							 & + \gamma_{2, 1} \cdot  k_{2,t} + \gamma_{2,2} \cdot  (k_{2,t})^2 + \gamma_{2,3} \cdot  \ind[k_{2,t} > 0] \\\nonumber
                                   & + \gamma_{2,4} \cdot  t + \gamma_{2,5} \cdot \ind[t < 18] \\\nonumber
                                  & + \gamma_{2,6} \cdot  \ind[a_{t-1} = 2]  + \gamma_{2,7} \cdot  k_{1,t} + \gamma_{2,8} \cdot  k_{3,t}
\end{align}
%
\noindent Equation (\ref{eq:UtilityBlueCollar}) shows the parameterization of the non-pecuniary utilities from working in a blue-collar occupation.
%
\begin{align}\label{eq:UtilityBlueCollar}
\zeta_{2}( \bm{k_t}, h_t, a_{t-1} ) = \,\alpha_2 & + c_{2,1} \cdot \ind[a_{t-1} \neq 2] + c_{2,2} \cdot \ind[k_{2,t} = 0]\\\nonumber
                            & + \vartheta_1 \cdot \ind[h_t \geq 12] + \vartheta_2 \cdot \ind[h_t \geq 16] + \vartheta_3 \cdot \ind[k_{3,t} = 1]
\end{align}
%-------------------------------------------------------------------------------
\subsubsection*{Military}
%-------------------------------------------------------------------------------
Equation (\ref{eq:SkillLevelMilitary}) shows the parameterization of the deterministic component of the skill production function.
%
\begin{align}\label{eq:SkillLevelMilitary}
    \Gamma_3( k_{3,t}, h_t, t, e_{j,3}) = e_{j,3} & + \beta_{3,1} \cdot h_t \\\nonumber
	               \nonumber &+ \gamma_{3,1} \cdot  k_{3,t} + \gamma_{3,2} \cdot (k_{3,t})^2 + \gamma_{3,3} \cdot \ind[k_{3,t} > 0]\\\nonumber
									 & + \gamma_{3,4} \cdot t + \gamma_{3,5} \cdot \ind[t < 18]
\end{align}
%
Contrary to the civilian sector there are no sheep-skin effects from graduation ($\beta_{3,2} = \beta_{3,3}= 0$). The previous occupational choice has no influence ($\gamma_{3,6}= 0$) and any experience other than military is non-transferable ($\gamma_{3,7} = \gamma_{3,8} = 0$).\\

\noindent Equation (\ref{eq:UtilityMilitary}) shows the parameterization of the non-pecuniary utility from working in the military.
%
\begin{align}\label{eq:UtilityMilitary}
\zeta_{3}( k_{3.t}, h_t)  = \,& c_{3,2} \cdot \ind[k_{3,t} = 0]+ \vartheta_1 \cdot \ind[h_t \geq 12] + \vartheta_2 \cdot \ind[h_t \geq 16]
\end{align}
%
Search costs $(c_{3, 1})$ are absent but there is a mobility cost if an individual has never served in the military before $c_{3,2}$. Individuals still experience a non-pecuniary utility from finishing high-school $\vartheta_1$ and college $\vartheta_2$.


\begin{Remark} Our parameterization for the per-period utility of serving in the military differs from \citet{Keane.1997} as we remain unsure about their exact specification. The authors state in Footnote 31 (p.498) that the constant for the non-pecuniary utility $\alpha_{3,t}$ depends on age. However, we are unable to determine the precise nature of the relationship. Equation (C3) (p.521) also indicates no productivity shock $\epsilon_{a,t}$ in the wage component. Table 7 (p.500) reports such estimates.
\end{Remark}
% ------------------------------------------------------------------------------
\FloatBarrier\subsection{Per-period utility for non-working alternatives}
% ------------------------------------------------------------------------------
The per-period utility for the non-working alternatives does not contain a wage component. Also, the productivity shock enters in an additive instead of a multiplicative fashion.
%-------------------------------------------------------------------------------
\subsubsection*{School}
%-------------------------------------------------------------------------------
Equation (\ref{eq:UtilitySchooling}) shows the parameterization of the non-pecuniary utility from schooling.
%
\begin{align}\label{eq:UtilitySchooling}
	\zeta_4(k_{3,t}, h_t, t, a_{t-1}, e_{j,4}, \epsilon_{4,t})  = e_{j,4} & + \beta_{tc_1} \cdot \ind[h_t \geq 12] + \beta_{tc_2} \cdot \ind[h_t \geq 16]   \\\nonumber
    							  & + \beta_{rc_1} \cdot \ind[a_{t-1} \neq 4, h_t < 12] + \beta_{rc_2} \cdot \ind[a_{t-1} \neq 4, h_t \geq 12] \\\nonumber
    							  & + \gamma_{4,4} \cdot t + \gamma_{4,5} \cdot \ind[t < 18] 																					  \\\nonumber
     							  & + \vartheta_1 \cdot \ind[h_t \geq 12] + \vartheta_2 \cdot \ind[h_t \geq 16] + \vartheta_3 \cdot \ind[k_{3,t} = 1]\\\nonumber
      							& + \epsilon_{4,t}
\end{align}
%
Costs include tuition and other fees for continuing schooling after high school $\beta_{tc_1}$ and college $\beta_{tc_2}$. Leaving the schooling alternative is reversible but returning entails adjustment costs that differ by schooling category ($\beta_{rc_1}, \beta_{rc_2}$). Once individuals reach a certain amount of schooling, they obtain a grade. There is no uncertainty about grade completion \citep{Altonji.1993} and there is no part-time enrollment. Individuals value completion of high-school and graduate school ($\vartheta_1, \vartheta_2$).
%-------------------------------------------------------------------------------
\subsubsection*{Home}
%-------------------------------------------------------------------------------
Equation (\ref{eq:UtilityHome}) shows the parameterization of the non-pecuniary utility from staying at home.
%
\begin{align}\label{eq:UtilityHome}
	\zeta_5(k_{3,t}, h_t, t, e_{j,5}, \epsilon_{5,1}) =  e_{j,5} & + \gamma_{5,4} \cdot \ind[18 \leq t \leq 20] + \gamma_{5,5} \cdot \ind[t \geq 21] \\ \nonumber
    							   & +\vartheta_{1} \cdot \ind[h_t \geq 12] + \vartheta_{2} \cdot \ind[h_t \geq 16] +  \vartheta_3 \cdot \ind[k_{3,t} = 1]  \\ \nonumber
    							   & + \epsilon_{5,t}
\end{align}
%
Staying at home as a young adult $\gamma_{5, 4}$ is less stigmatic as doing so while already being an adult $\gamma_{5,5}$. Additionally, possessing a degree  $(\vartheta_1, \vartheta_2)$ or leaving the military prematurely $\vartheta_3$ influences the per-period utility.
%-------------------------------------------------------------------------------
\FloatBarrier\subsection{Overview parameters}
%-------------------------------------------------------------------------------

\begin{ThreePartTable}
% Information available at https://ftp.agdsn.de/pub/mirrors/latex/dante/macros/latex/contrib/threeparttablex/threeparttablex.pdf

\begin{TableNotes}
	\item \textbf{Note:} The listed parameters are represented as an overview. The per-period utilities for the alternatives do not necessarily include all of them.
\end{TableNotes}

\begin{longtable}{@{}cll@{}}
\caption{Overview of parameters in the \citet{Keane.1997} extended model.}
\label{tab:ModelParameters}

\setlength\extrarowheight{2.5pt}

% Settings longtable
\\
\toprule 
\textbf{Parameter}            &  &  \multicolumn{1}{l}{\textbf{Description}}              \\ \midrule 
\endfirsthead 

% Alternative 1 header for the beginning on next page
%\midrule
%Parameter            &  &  \multicolumn{1}{l}{Description}     \\ \midrule

% Alternative 2 header for the beginning on next page
\multicolumn{3}{c}{\tikz\draw [thick,dash dot] (0,0) -- (15,0);} \vspace{-5pt} \\
\multicolumn{3}{c}{continued from previous page} \vspace{-10pt} \\
\multicolumn{3}{c}{\tikz\draw [thick,dash dot] (0,0) -- (15,0);} \\
\endhead 

\multicolumn{3}{c}{\tikz\draw [thick,dash dot] (0,0) -- (15,0);} \vspace{-5pt} \\
\multicolumn{3}{c}{continued on next page } \vspace{-10pt} \\
\multicolumn{3}{c}{\tikz\draw [thick,dash dot] (0,0) -- (15,0);} \\
\endfoot

\bottomrule 
\insertTableNotes
\endlastfoot 

% Start table
\midrule
\multicolumn{3}{l}{Preference and type-specific parameters}																		 \\ \midrule 
$\delta$ 				&  & discount factor																			  \\
$e_{j, a}$			&  & initial endowment of type $j$ in alternative $a$ specific skills 	  \\ [7.5pt] \midrule 
\multicolumn{3}{l}{Common parameters per-period utility}												\\ \midrule 
$\alpha_a$           &  & return non-wage working conditions		   \\
$\vartheta_1$        &  & non-pecuniary premium of finishing high-school                 								    \\
$\vartheta_2$        &  & non-pecuniary premium finishing college															    \\
$\vartheta_3$        &  & non-pecuniary premium of leaving military early						  \\[7.5pt] \midrule 
\multicolumn{3}{l}{Schooling-related parameters}															   \\ \midrule
% Work
$\beta_{a,1}$        &  & return additional year of completed schooling 								\\
$\beta_{a,2}$        &  & skill premium high-school graduate										      \\
$\beta_{a,3}$        &  & skill premium college graduate													   	\\
% Military
% School
$\beta_{tc_1}$       &  & tuition cost high-school                      											\\
$\beta_{tc_2}$       &  & tuition cost college                          												\\
$\beta_{rc_1}$       &  & re-entry cost high-school                     										   \\
$\beta_{rc_2}$       &  & re-entry cost college                        												   \\
% Home
$\beta_{5,2}$        &  & skill premium high-school graduate            									\\
$\beta_{5,3}$        &  & skill premium college graduate                									     \\ [7.5pt] \midrule
\multicolumn{3}{l}{Experience-related parameters}           													 \\
\midrule 
$\gamma_{a,1}$       &  & return same-sector experience                 									 \\
$\gamma_{a,2}$       &  & return squared same-sector experience         								\\
$\gamma_{a,3}$       &  & premium having worked in sector before        							   \\
$\gamma_{a,4}$       &  & return age effect                             											     \\
$\gamma_{a,5}$       &  & return age effect being a minor               										\\
$\gamma_{a,6}$       &  & premium remaining in same sector              								   \\
$\gamma_{a,7}$       &  & return civilian cross-sector experience       								    \\
$\gamma_{a,8}$       &  & return non-civilian sector experience       										 \\
$\gamma_{3,1}$       &  & return same-sector experience                 									  \\
$\gamma_{3,2}$       &  & return squared same-sector experience    										 \\
$\gamma_{3,3}$       &  & premium having worked in sector before   										\\
$\gamma_{3,4}$       &  & return age effect                             												 \\
$\gamma_{3,5}$       &  & return age effect being a minor              	   										\\
$\gamma_{4,4}$       &  & return age effect                             												 \\
$\gamma_{4,5}$       &  & return age effect being a minor                  										\\
$\gamma_{5,4}$       &  & return age between 17 and 21                 	  									   \\
$\gamma_{5,5}$       &  & return age older than 20							   										\\[7.5pt] \midrule
\multicolumn{3}{l}{Mobility and search parameters}          													  \\ \midrule 
$c_{a,1}$            &  & premium switching to occupation $a$           									   \\
$c_{a,2}$            &  & premium for working first time in occupation $a$         										  \\
$c_{3,2}$            &  & premium for serving first time in military    										  \\[7.5pt] \midrule
\multicolumn{3}{l}{Error correlation}          													  									\\ \midrule 
$\sigma_{a,a}$	&	& standard deviation of shock in alternative $a$									\\
$\sigma_{i,j}$ &	& correlation between shocks of alternative $a = i$ and $a=j$ with $i \neq j$ \\

\end{longtable}
\end{ThreePartTable}

